\documentclass{article} % For LaTeX2e
\usepackage{nips12submit_e,times}
%\documentstyle[nips12submit_09,times,art10]{article} % For LaTeX 2.09

\usepackage{amsmath, amsthm, amssymb}
% \usepackage{natbib}
\usepackage{graphicx}
\usepackage{wrapfig}
\usepackage{subcaption}

\pagestyle{empty}

\title{Mental Rotation as Bayesian Quadrature}
\author{Jessica B.~Hamrick and Thomas L.~Griffiths\\
  Department of Psychology, University of California, Berkeley\\
  Berkeley, CA 94720\\
  \texttt{\{jhamrick,tom\_griffiths\}@berkeley.edu}}

% The \author macro works with any number of authors. There are two commands
% used to separate the names and addresses of multiple authors: \And and \AND.
%
% Using \And between authors leaves it to \LaTeX{} to determine where to break
% the lines. Using \AND forces a linebreak at that point. So, if \LaTeX{}
% puts 3 of 4 authors names on the first line, and the last on the second
% line, try using \AND instead of \And before the third author name.

\newcommand{\TODO}[1]{\textcolor{red}{[TODO: #1]}}
\newcommand{\ME}[0]{\mathrm{ME}}
\newcommand{\naive}[0]{na\"ive}
\newcommand{\Naive}[0]{Na\"ive}

%% DO NOT EDIT MANUALLY
%% This file was automatically generated by code/analysis.ipynb
%% Rerun that notebook if you want to update this file.

%% Corr
\newcommand{\Naivecorr}[0]{\rho=0.82}
\newcommand{\Naivecorrsm}[0]{\rho=0.99}
\newcommand{\Naivecorrbg}[0]{\rho=-0.56}
\newcommand{\VMcorr}[0]{\rho=0.91}
\newcommand{\VMcorrsm}[0]{\rho=0.99}
\newcommand{\VMcorrbg}[0]{\rho=-0.01}
\newcommand{\BQcorr}[0]{\rho=0.98}
\newcommand{\BQcorrsm}[0]{\rho=0.99}
\newcommand{\BQcorrbg}[0]{\rho=0.88}

%% ME
\newcommand{\GSME}[0]{\ME{}=0.00}
\newcommand{\GSMEdiff}[0]{\ME{}=0.00}
\newcommand{\GSMEsame}[0]{\ME{}=0.00}
\newcommand{\NaiveME}[0]{\ME{}=0.18}
\newcommand{\NaiveMEdiff}[0]{\ME{}=0.29}
\newcommand{\NaiveMEsame}[0]{\ME{}=0.08}
\newcommand{\VMME}[0]{\ME{}=0.16}
\newcommand{\VMMEdiff}[0]{\ME{}=0.12}
\newcommand{\VMMEsame}[0]{\ME{}=0.19}
\newcommand{\BQME}[0]{\ME{}=0.04}
\newcommand{\BQMEdiff}[0]{\ME{}=0.05}
\newcommand{\BQMEsame}[0]{\ME{}=0.04}

%% MSE
\newcommand{\NaiveMSE}[0]{\MSE{}=0.030}
\newcommand{\VMMSE}[0]{\MSE{}=0.002}
\newcommand{\BQMSE}[0]{\MSE{}=0.001}



\nipsfinalcopy % Uncomment for camera-ready version

\begin{document}

\maketitle

\begin{abstract}
  Given a computational resource--for example, the ability to
  visualize an object rotating--how do you best make use of it? We
  explored how mental simulation should be used in the classic
  psychological task of determining if two images depict the same
  object in different orientations. We compared two models on this
  mental rotation task, and found that a model based on an optimal
  experiment design for Bayesian quadrature is qualitatively more
  consistent with classic behavioral data than a simpler model. We
  suggest that rational models which adaptively exploit available
  resources are promising in their ability to characterize
  metacognitive processes like mental simulation.
\end{abstract}

\section{Introduction}

One of the challenges of solving any computational problem is
determining how best to use the available computing resources. For
example, a computer can render complex graphics faster by recognizing
that this kind of computation should be carried out by a specialized
graphics processor. The same challenge arises in designing an
intelligent agent: how should the agent make best use of its computing
resources? Recent research on rational models of human cognition has
provided insight into the nature of the computational problems that
human beings need to solve (e.g., \cite{Chater:1999wp,tenenbaumkgg11}),
but leaves open the question of how people allocate their resources in
solving those problems. In this paper, we take a step towards
addressing this question, applying rational analysis (in the spirit of
\cite{Marr:1983to,anderson90,Shepard:1987tt}) to one aspect of human
metacognition: the use of mental simulation.

Consider the images on the left in Fig.~\ref{fig:mental-rotation}. In
each panel, are the two depicted objects identical (except for a
rotation), or distinct? When presented with this mental rotation task,
people default to a strategy in which they visualize one object
rotating until it is congruent with the other
\cite{Shepard1971}. There is strong evidence for such ``mental
simulation'': we can imagine three-dimensional objects in our minds
and manipulate them, to a certain extent, as if they were real
\cite{Kosslyn:2009tj}.  However, the use of mental simulation is
predicated on determining appropriate parameters to give the
simulation, analogous to determining exactly what computation should
be passed to a graphics processor.  In the case of the classic mental
rotation task, we might ask: How do people know which way to rotate
the object?  When should one stop rotating and accept the hypothesis
that the objects are different?

Recent work in cognitive science has shown how the allocation of
cognitive resources to solving computational problems can be analyzed
using the methods of statistical decision theory
\cite{Lieder:2012wg,Vul:2009wy}. We suggest that this ``rational
metacognition'' approach may also be applied to the problem of mental
rotation. Specifically, we hypothesize that mental rotation can be
framed as integration over a probability distribution, with the
direction of rotation becoming an optimal experiment design problem
(or in machine learning parlance, an active learning problem). In an
initial investigation into this hypothesis, we find that recent
methods for Bayesian quadrature
\cite{Diaconis:1988uo,OHagan:1991tx,Osborne:2012tm}, in contrast to a
simpler heuristic model, provide a possible solution to determining
the direction and extent of rotation.

\begin{figure}[t]
  \centering
  \includegraphics[width=0.9\textwidth]{../figures/shepard-rotation.png}
  \caption{\textbf{Classic mental rotation task}. Participants in
    \cite{Shepard1971} saw stimuli such as those on the left, and
    judged whether each pair of shapes was the same shape in two
    different orientations (``same'' pairs), or two different shapes
    (``different'' pairs). \textbf{A} and \textbf{B} show ``same''
    pairs, while \textbf{C} shows a ``different'' pair. The plots on
    the right indicate mean response times on the mental rotation
    task, exhibiting a strong linear relationship with increasing
    variance as a function of true rotation.}
  \label{fig:mental-rotation}
\end{figure}

\section{Computational-level model}

We begin by analyzing mental rotation at Marr's \textit{computational}
level \cite{Marr:1983to}: what is the problem to be solved?  Formally,
people are presented with two images, $X_a$ and $X_b$, which are the
coordinates of the vertices of 2D shapes (e.g.,
Fig.~\ref{fig:stimuli}). Participants must determine whether $X_a$ and
$X_b$ depict the same shape, i.e., whether $\exists R\textrm{ s.t. }
X_b=RX_a$, where $R$ is a rotation matrix. We can formulate the
judgment of whether $X_a$ and $X_b$ have the same origins by deciding
about two hypotheses, $h_0$: $\forall R\ X_b\neq RX_a$ and $h_1$:
$\exists R\textrm{ s.t. } X_b=RX_a$.  To compare the hypotheses, we
need to compute the posterior for each: $p(h\ \vert\ X_a, X_b)\propto
p(X_a, X_b\ \vert\ h)p(h)$. Assuming the hypotheses are equally likely
\textit{a priori}, the prior term $p(h)$ will cancel out when
comparing $h_0$ and $h_1$, thus allowing us to focus on the
likelihoods, which are $p(X_a,\ X_b\ \vert \ h_0)=p(X_a)p(X_b)$ for
$h_0$ and $p(X_a,\ X_b\ \vert \ h_1)=\int_R p(X_a) p(X_b\vert X_a,R)
p(R)\ \mathrm{d}R$ for $h_1$. From these likelihoods, we compute the
ratio $\ell$ which is given by $\ell=\left(\int_R p(X_b\ \vert\ X_a,
  R)p(R)\ \mathrm{d}R\right) /\ p(X_b)$. If $\ell<1$, then $h_0$ is
the more likely hypothesis. If $\ell>1$, then $h_1$ is the more likely
hypothesis.

\section{Algorithmic approximation}

We define the prior probability of shape $X$ to be
$p(X)=n!\left(\frac{1}{2\pi}\right)^n$ according to a generative
procedure.\footnote{A set of $n$ vertices could be chosen in any of
  $n!$ different ways, and each vertex is located at a random angle
  (between 0 and $2\pi$) and radius (between 0 and 1).} This gives us
the denominator of $\ell$. Computing the numerator is more difficult,
as we we cannot compute $p(X_b\vert X_a, R)$ directly. Instead, we
introduce a new variable $X_R$ denoting a mental image, which
approximates $RX_a$. The $X_R$ are generated by repeated application
of a function $\tau$, i.e. $X_R=RX_a=\tau(X_{R-r},
r)=\tau(\tau(X_{R-2r}, r), r)=\ldots{}=\tau^{(\frac{R}{r})}(X_a, r)$,
where $r$ is a small angle, and $\tau^{(i)}$ indicates $i$ recursive
applications of $\tau$. Using this sequential function, we get:
\begin{align}
  &p(X_a, X_b\ \vert \ h_1)=\int_R \int_{X} p(X_b\vert X) p(X\vert X_a, R)p(X_a)p(R)\ \mathrm{d}X\ \mathrm{d}R \nonumber \\
  &= \int_R \int_X p(X_b\vert X)\delta(\tau^{(\frac{R}{r})}(X_a, r)-X)p(X_a)p(R)\ \mathrm{d}X\ \mathrm{d}R = \int_R p(X_b\vert X_R)p(X_a)p(R)\ \mathrm{d}R
\end{align}
However, the exact form of $p(X_b\vert X_R)$ is still unknown. We
approximate it with a similarity function $S(X_b, X_R)$, and denote
the resulting integral as $Z=\int_R S(X_b, X_R)p(R)\
\mathrm{d}R\approx \int_R p(X_b\vert X_R)p(R)\ \mathrm{d}R$.  We
define the similarity based on Gaussian similarity and possible
mappings $M$ of the vertices,\footnote{Because the vertices are
  connected in a way which forms a closed loop, we need only consider
  $2n$ mappings of the $n$ vertices (we assume uncertainty for which
  is the ``first'' vertex, and then which of its two neighbors is the
  ``second''). So, the possible orderings are of the form
  $M=\lbrace{}0, 1, \ldots{}, n\rbrace{}$, $M=\lbrace{}n, 0, \ldots{},
  n-1\rbrace{}$, etc.} i.e. $S(X_b,
X_R)=\frac{1}{2n}\sum_{M\in\mathbb{M}}\prod_{i=1}^n\mathcal{N}(X_b[i]\
\vert \ (MX_R)[i], \Sigma)$ where $i$ denotes the $i^{th}$ vertex. An
example stimulus and corresponding $S$ is shown in
Fig.~\ref{fig:shapes}.

To summarize, the process of generating a mental image consists of
computing a single $X_R$ and then computing $S(X_b, X_R)$. We denote
the sequence of rotations computed by this procedure as
$\mathbf{R}=\{R_1, R_2, \ldots{}\}$. However, this sequence cannot be
arbitrary, as mental rotation is computationally demanding. Our goal
is to minimize the number of rotations $\vert\mathbf{R}\vert$ while
still obtaining an estimate of $Z$ that is accurate enough to choose
the correct hypothesis.

\paragraph{\Naive{}}

As a lower bound on performance, we defined a \naive{} model which
performs a hill-climbing search over the similarity function until it
reaches a (possibly local) maximum. Once a maximum as been found, the
model computes an estimate of $Z$ by linearly interpolating between
sampled rotations (e.g., Fig.~\ref{fig:li}).

\paragraph{Bayesian Quadrature}

A more flexible strategy uses what is known as \emph{Bayesian
  Quadrature} (BQ) \cite{Diaconis:1988uo,OHagan:1991tx} to estimate
$Z$.  BQ computes a posterior distribution over $Z$ by placing a
Gaussian Process (GP) prior on the function $S$ and evaluating $S$ at
a particular set of points. However, while $S$ is a non-negative
likelihood function, GP regression enforces no such
constraint. \cite{Osborne:2012tm} give a method to place a prior over
the \textit{log} likelihood, thus ensuring that $S=e^{\log S}$ will be
positive, i.e.  $E[Z\ \vert \ \log S]=\int_{\log S}\left(\int_R
  \exp(\log{S(X_b,X_R)})p(R)\
  \mathrm{d}R\right)\mathcal{N}\left(\log{S}\ \vert \ \mu_{\log S},
  \Sigma_{\log S}\right)\ \mathrm{d}\log S$, where $\mu_{\log S}$ and
$\Sigma_{\log S}$ are the mean and covariance, respectively, of the GP
over $\log S$ given $\mathbf{R}$. We approximate this according to the
method given in \cite{Osborne:2012tm}, i.e. $\mu_Z=E[Z\ \vert \ S,
\log S, \Delta_c] \approx \int_R \mu_{S}(1 + \mu_{\Delta_c}) p(R)\
\mathrm{d}R$, where $\mu_S$ is the mean of a GP over $S$ given
$\mathbf{R}$; and $\mu_{\Delta_c}$ of a GP over $\Delta_c=\mu_{\log S}
- \log \mu_S$ given $\mathbf{R}_c$, which consists of $\mathbf{R}$ and
a set of intermediate \emph{candidate points} $c$ as described in
\cite{Osborne:2012tm}. The variance is $\tilde{V}(Z\vert S, \log S,
\Delta_c)$ as defined in Equation 12 of \cite{Osborne:2012tm}.


\begin{figure}[t]
  \centering
  \begin{subfigure}[b]{0.32\textwidth}
    \centering
    \includegraphics[width=\textwidth]{../figures/stimuli_shapes.pdf}
    \vspace{0pt}
    \caption{Example stimuli}
    \label{fig:stimuli}
  \end{subfigure}
  \begin{subfigure}[b]{0.32\textwidth}
    \centering
    \includegraphics[width=\textwidth]{../figures/li_regression.pdf}
    \caption{\Naive{} model}
    \label{fig:li}
  \end{subfigure}
  \begin{subfigure}[b]{0.32\textwidth}
    \centering
    \includegraphics[width=\textwidth]{../figures/bq_regression_final.pdf}
    \caption{Bayesian Quadrature Model}
    \vspace{0pt}
    \label{fig:bq}
  \end{subfigure}
  \caption{\textbf{Example model behavior.} \textbf{(a)} An example
    stimulus in which the shapes differ only by a rotation. All
    stimuli consist of three to six vertices centered around the
    origin, and edges which create a closed loop from the
    vertices. The true angle of rotation between $X_a$ and $X_b$ is at
    $\frac{2\pi}{3}$. \textbf{(b-c)} Likelihood function and \naive{}
    (b) and BQ (c) model estimates. The sampled points $\mathbf{R}$
    (red circles) are then used to estimate $S$ (black lines are the
    true $S$, red lines are the estimate).}
  \label{fig:shapes}
\end{figure}

We pick the initial direction of rotation which results in the higher
value of $S$. From then on, at each step we compute $\mu_Z$ and
$\tilde{V}$ to estimate a distribution over the likelihood ratio
$\ell$, i.e.  $p(\ell)\approx\frac{1}{p(X_b)}\ \mathcal{N}(Z\ \vert\
\mu_Z, \sigma_z)$.  We choose $h_0$ when $p(\ell < 1)\geq 0.95$, and
$h_1$ when $p(\ell > 1)\geq 0.95$.\footnote{We chose a threshold of
  0.95 because the standard confidence interval is 95\%. In the
  future, however, this threshold could be fit to human data.} Until
one of these conditions are met (or the shape has been fully rotated),
the model will continue to compute rotations and update its estimate
of $Z$.  Additionally, the model will change direction if doing so
would lower the \textit{expected} posterior variance of $Z$ given some
new sample $a$.\footnote{This is similar to the full posterior
  variance calculation given in \cite{Osborne:2012tm}, however we
  compute the variance given only the current mean estimate of $a$.}
Thus, it is able to actively change its strategy, unlike the
hill-climbing procedure.

\section{Results}

We evaluated each model's performance on 20 randomly generated shapes
which had between three and six vertices, inclusive, (e.g., $X_a$ in
Fig.~\ref{fig:stimuli}). For each shape, we computed 18 ``same'' and
18 ``different'' stimuli pairs, with $R$ spaced at $20^\prime$
increments between 0 and 360, as in \cite{Shepard1971}. ``Same'' pairs
were created by rotating $X_a$ by $R$; the same was true for
``different'' pairs, except that $X_a$ was also reflected across the
$y$-axis. To gauge performance, we looked at \textit{response error
  rates}: how accurate was the model at choosing the correct
hypothesis? This was defined as the mean error ($\ME{}$), or fraction
of times the model chose incorrectly.  We additionally looked at
\textit{rotations}: for those ``same'' pairs which the model judged
correctly, how correlated were the model's rotations with the true
angles of rotation?  We quantified this using the Pearson's
correlation coefficient $\rho$ for the true rotation, $R$, versus the
number of steps/rotations take by the model, $\vert
\mathbf{R}\vert$. Fig.~\ref{fig:rotations} shows true rotations vs
(\ref{fig:all-stimuli}) the number of steps/rotations taken by the
model for individual stimuli, and (\ref{fig:correct-stimuli}) the
average number of steps taken for each true rotation. This latter
analysis can be qualitatively compared to the results of
\cite{Shepard1971}, as shown in Fig.~\ref{fig:mental-rotation}.

\paragraph{\Naive{}} 

The \naive{} model's response error rate was $\NaiveME{}$, which is
better than chance (where chance is equivalent to guessing randomly,
i.e. $\ME{}=0.5$). The correlation between the \naive{} model's
average rotation and the true angle of rotation was $\Naivecorr{}$
(Fig.~\ref{fig:correct-stimuli}, left). As shown in
Fig.~\ref{fig:correct-stimuli} (right), the \naive{} model corresponds
extremely well to the true angle of rotation when $R<\frac{\pi}{2}$;
this is because it needs to rotate less and is therefore less likely
to get stuck on local maxima. For $R>\frac{\pi}{2}$, we see an
increasing tendency to under-rotate due to getting stuck on local
maxima, as well as a tendency to over-rotate if the wrong direction
was initially chosen.

\paragraph{Bayesian Quadrature}

The BQ model was much more accurate in choosing the correct hypothesis
than the \naive{} model ($\BQME{}$). The number of rotations computed
by the BQ model were strongly correlated with the true rotations
($\BQcorr$), a result which is qualitatively similar to that exhibited
by humans (Fig.~\ref{fig:mental-rotation}
vs. Fig.~\ref{fig:correct-stimuli}, right). Because the BQ model has
the capacity to ``reset'', it could recover from rotating in the
incorrect direction (e.g., Fig.~\ref{fig:bq}) and thus did not
over-rotate as frequently as the \naive{} model.  The BQ model also
under-rotated less frequently because it rotates until it is confident
in its estimate of $Z$ and thus does not get stuck on local optima.

\begin{figure}[t]
  \centering
  \begin{subfigure}[b]{0.46\textwidth}
    \centering
    \includegraphics[width=\textwidth]{../figures/model_rotations_A.pdf}
    \caption{\textbf{All stimuli}. Each subplot shows the
      correspondence between the true angle of rotation ($R$) for
      ``same'' pairs and the amount of rotation performed by the
      model.}
    \label{fig:all-stimuli}
  \end{subfigure}
  \hspace{0.05\textwidth}
  \begin{subfigure}[b]{0.46\textwidth}
    \centering
    \includegraphics[width=\textwidth]{../figures/model_rotations_B.pdf}
    \caption{\textbf{Correct stimuli}. Each subplot shows the models'
      mean rotations over stimuli pairs that were judged
      correctly. Black dots correspond to ``same'' pairs, and blue
      lines to ``different'' pairs.}
    \label{fig:correct-stimuli}
  \end{subfigure}
  \caption{\textbf{Model rotations.} Error bars/shaded regions
    indicate one standard deviation, and the dotted lines indicate the
    least-squares fit to the ``same'' pairs.}
  \label{fig:rotations}
\end{figure}


\section{Conclusion}

How do people allocate their mental simulation resources?  We
performed an initial investigation into the specific case of mental
rotation, using rational analysis to characterize optimal strategies.
We demonstrated that the classic mental rotation task
\cite{Shepard1971} presents a non-trivial computational problem and
cannot be solved with a simple, heuristic-based model. In contrast, an
adaptive, Bayesian Quadrature model provides answers to puzzling
questions surrounding the incremental nature of mental rotation: which
way should the object be rotated, and for how long? This model
formalizes these answers in a way that is qualitatively consistent
with human behavior, both in response time linearity
\cite{Shepard1971} and variability \cite{Just1976}. Although this
research is still in its first stages, these initial results support
the idea that mental rotation may be another instance in which people
do appropriately use available computational resources to solve the
task at hand.

\paragraph{Acknowledgments} This research was supported by ONR MURI
grant number N00014-13-1-0341, and a Berkeley Fellowship awarded to
JBH.

\vfill


% \renewcommand\bibsection{\subsubsection*{\refname}}
\renewcommand\refname{\normalsize{References}}
\bibliographystyle{ieeetr}
{\small \bibliography{references}}

\end{document}



