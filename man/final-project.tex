\documentclass{article} % For LaTeX2e
\usepackage{nips12submit_e,times}
%\documentstyle[nips12submit_09,times,art10]{article} % For LaTeX 2.09

\usepackage{amsmath, amsthm, amssymb}
% \usepackage{natbib}
\usepackage{graphicx}
\usepackage{wrapfig}
\usepackage{subcaption}

\pagestyle{empty}

\title{Optimal strategies in mental rotation tasks}
\author{Jessica B.~Hamrick\\
  Department of Psychology\\
  University of California, Berkeley\\
  Berkeley, CA 94720\\
  \texttt{jhamrick@berkeley.edu}}

% The \author macro works with any number of authors. There are two commands
% used to separate the names and addresses of multiple authors: \And and \AND.
%
% Using \And between authors leaves it to \LaTeX{} to determine where to break
% the lines. Using \AND forces a linebreak at that point. So, if \LaTeX{}
% puts 3 of 4 authors names on the first line, and the last on the second
% line, try using \AND instead of \And before the third author name.

\newcommand{\TODO}[1]{\textcolor{red}{[TODO: #1]}}

\nipsfinalcopy % Uncomment for camera-ready version

\begin{document}

\maketitle

\begin{abstract}
\TODO{}
\end{abstract}

\section{Introduction}

% 1. The big picture
% 2. Our contribution
% 3. How we do it
% 4. What's in the paper

Consider the objects in Figure \ref{fig:mental-rotation}. In each
panel, are the two depicted objects identical (except for a rotation),
or distinct? When presented with this mental rotation task, people
default to a strategy in which they visualize one object rotating
until it is congruent with the other \cite{Shepard1971}. There is
strong evidence for such \textit{mental imagery} or \textit{mental
  simulation}: we can imagine three-dimensional objects in our minds
and manipulate them, to a certain extent, as if they were real
\cite{Kosslyn:2009tj}.

However, the use of mental simulation is predicated on determining
appropriate parameters to give the simulation, and people's cognitive
constraints may furthermore place limitations on the duration or
precision of simulation. One hypothesis for how these issues are
handled argues that people use a ``rational'' solution, meaning that
it is optimal under the given constraints
\cite{Lieder:2012wg,Vul:2009wy,Griffiths2012a}.

In the case of the classic mental rotation task, we might ask: in what
direction should the object be rotated?  What are the requirements for
``congruency''? When should one stop rotating and accept the
hypothesis that the objects are different? In this project, I will
investigate the optimal computational solution to this task. Thus, the
specific question this project aims to answer is: what is the rational
computational solution to the mental rotation task, and does it
predict the people's behavior on the task?

\section{Background}

% We do not know the form of $S$, and so cannot compute this integral
% analytically. We cannot rely on a simple Monte Carlo simulation to
% estimate it, either: because mental rotation is costly in terms of
% cognitive resources and must be performed in a sequence, we cannot
% evaluate $S(I_b|I_M)$ at arbitrary $I_M$. Instead, we must choose a
% relatively small, sequential set of $I_M$ to estimate the
% integral. 

% Bayesian quadrature \cite{Diaconis:1988uo} provides an alternate
% method for evaluating the integral by placing a prior distribution on
% functions and then computing a posterior over values of the
% integral. While Bayesian quadrature itself does not address the issue
% of choosing points to evaluate the function at, recent work in
% statistics and machine learning has examined how to optimally select
% these samples \cite{Osborne:2012tm}.

\begin{figure}[t]
  \centering
  \includegraphics[width=0.95\textwidth]{../figures/shepard-rotation.png}
  \caption{Classic mental rotation task from \cite{Shepard1971}.}
  \label{fig:mental-rotation}
\end{figure}

\section{Computational model}

\begin{figure}[t]
  \centering
  \begin{subfigure}[b]{0.45\textwidth}
    \centering
    \includegraphics[width=\textwidth]{../figures/stimuli_shapes.pdf}
    \vspace{0pt}
    \caption{Example stimuli}
    \label{fig:stimuli}
  \end{subfigure}
  \begin{subfigure}[b]{0.45\textwidth}
    \centering
    \includegraphics[width=\textwidth]{../figures/likelihood_function.pdf}
    \caption{Likelihood function}
    \label{fig:likelihood}
  \end{subfigure}
  \caption{Rotated shapes and corresponding similarities.}
\end{figure}

People are presented with two images, $X_a$ and $X_b$, which are the
coordinates of the vertices of two shapes. Participants must determine
whether $X_a$ and $X_b$ were generated from the same (albeit possibly
transformed and permuted) original shape, i.e., whether $\exists
R,M\textrm{ s.t. } X_b=MRX_a$, where $M$ is a permutation matrix and
$R$ is a rotation matrix. \TODO{Figure with some example shapes}

We can formulate the judgment of whether $X_a$ and $X_b$ have the same
origins by deciding about two hypotheses:

\begin{itemize}
\itemsep1pt\parskip0pt\parsep0pt
\item
  $h_0$: $\forall M,R\ X_b\neq MRX_a$
\item
  $h_1$: $\exists M,R\textrm{ s.t. } X_b=MRX_a$
\end{itemize}

To compare the hypotheses, we need to compute the likelihood of each
hypothesis, i.e. $p(X_a,\ X_b\ \vert \ h_0)$ and $p(X_a,\ X_b\ \vert \
h_1)$.

The likelihood is easy to compute under $h_0$, because we assume that
$X_a$ and $X_b$ are independent. Thus:
\begin{equation}
  p(X_a,\ X_b\ \vert \ h_0)=p(X_a)p(X_b)
  \label{eq:lh-h0}
\end{equation}

When the configurations are the same, the likelihood becomes more
complicated:
\begin{equation} 
  p(X_a,\ X_b\ \vert \ h_1)=\int_R\int_M p(X_a) p(X_b\vert X_a,R,M) p(R) p(M)\ \mathrm{d}M\ \mathrm{d}R
\end{equation}

For a small number of vertices, we can compute the integral over $M$
by enumerating every possible mapping between $X_a$ and $X_b$. After
doing so, we obtain:
\begin{equation} 
  p(X_a,\ X_b\ \vert \ h_1)=\int_R p(X_a) p(X_b\vert X_a,R) p(R)\ \mathrm{d}R
\end{equation}

However, we cannot compute $p(X_b\vert X_a, R)$ directly. Instead, we
introduce a new variable $X_R$ denoting a mental image, which
approximates $RX_a$. The $X_R$ are generated sequentially by repeated
application of a function $\tau$:
\begin{align}
  X_R&=RX_a\nonumber \\
  &=\tau(X_{R-r}, r)\nonumber \\
  &=\tau(\tau(X_{R-2r}, r), r)\nonumber \\
  &\ldots{}\nonumber \\
  &=\tau^{(\frac{R}{r})}(X_a, r)
  \label{eq:tau}
\end{align} 
Where $r$ is a small angle, and $\tau^{(i)}$ indicates $i$ recursive
applications of $\tau$. Using this sequential function, we get:
\begin{align}
  p(X_a, X_b\ \vert \ h_1)&=\int_R \int_{X} p(X_b\vert X) p(X\vert X_a, R)p(X_a)p(R)\ \mathrm{d}X\ \mathrm{d}R \nonumber \\
  x&= \int_R \int_X p(X_b\vert X)\delta(\tau^{(\frac{R}{r})}(X_a, r)-X)p(X_a)p(R)\ \mathrm{d}X\ \mathrm{d}R \nonumber \\
  &= \int_R p(X_b\vert X_R)p(X_a)p(R)\ \mathrm{d}R
  \label{eq:lh-h1}
\end{align}

Once we have computed both likelihoods, we compute their ratio:
\begin{equation}
  \ell=\frac{p(X_a, X_b\ \vert \ h_1)}{p(X_a, X_b\ \vert \ h_0)}=\frac{\int_R p(X_b\ \vert\ X_R)p(R)\ \mathrm{d}R}{p(X_b)}
  \label{eq:lh-ratio}
\end{equation}
If $\ell<1$, then $h_0$ is the more likely hypothesis. If $\ell>1$,
then $h_1$ is the more likely hypothesis.

\section{Implementation}

We define the prior probabilities over stimuli based on how they are
generated. For shapes with $n$ vertices, each vertex is at a random
angle with a radius random chosen between 0 and 1, in polar
coordinates, and could be chosen in any of $n!$ different ways. Thus,
the prior over a shape $X$ is:
\begin{equation}
  p(X)=n!\left(\frac{1}{2\pi}\right)^n
  \label{eq:prior}
\end{equation} 
which gives us the denominator in Equation \ref{eq:lh-ratio}. 

Computing the numerator of Equation \ref{eq:lh-ratio} is more
difficult, as we do know $p(X_b\vert X_R)$. We approximate it with a
similarity function $S(X_b, X_R)$, which also takes into account the
different possible mappings of vertices:
\begin{equation}
  Z=\int_R S(X_b, X_R)p(R)\ \mathrm{d}R\ \approx\int_Rp(X_b\ \vert\ X_R)p(R)\ \mathrm{d}R
  \label{eq:Z}
\end{equation}

Because the vertices are connected in a way which forms a closed loop,
we need only consider $n$ mappings, $M$, of the $n$ vertices (we
assume the uncertainty is only in which is the ``first''
vertex). Thus, the possible orderings are of the form $M=\lbrace{}0,
1, \ldots{}, n\rbrace{}$, $M=\lbrace{}n, 0, \ldots{}, n-1\rbrace{}$,
and so on. This gives us an explicit form for the similarity function:
\begin{equation}
  S(X_b, X_R)=\frac{1}{n}\sum_{M\in\mathbb{M}}\prod_{i=1}^n\mathcal{N}(X_b[i]\ \vert \ (MX_R)[i], \Sigma)
  \label{eq:similarity}
\end{equation}
where $i$ denotes the $i^{th}$ vertex. An example of $S$ for the
stimuli shown in Figure \ref{fig:stimuli} is illustrated in Figure
\ref{fig:likelihood}.

This process of mental rotation (i.e., generating a single $X_R$ (as
in Equation \ref{eq:tau}) and then computing $S(X_b, X_R)$) is a
computationally demanding cognitive process. Thus, our goal is to
minimize the number of rotations while still obtaining an estimate of
$Z$ that is accurate enough to choose the correct hypothesis. With
this in mind, we now examine several approaches to estimating $Z$.

\subsection{Gold standard model}

To compare the accuracy of other models' estimates of $Z$, we computed
a ``gold standard''\footnote{This only gives an accurate estimate of
  $Z$, which is itself an approximation, and is thus not necessarily
  the true value of the numerator in Equation \ref{eq:lh-ratio}.} by
evaluating $S(X_b, X_R)$ at 360 values of $R$ spaced evenly between
$0$ and $2\pi$.

\subsection{Na\"ive model}

\begin{figure}[t]
  \centering
  \begin{subfigure}[b]{0.45\textwidth}
    \centering
    \includegraphics[width=\textwidth]{../figures/li_regression.pdf}
    \caption{Linear interpolation}
    \label{fig:li}
  \end{subfigure}
  \begin{subfigure}[b]{0.45\textwidth}
    \centering
    \includegraphics[width=\textwidth]{../figures/vm_regression.pdf}
    \caption{Parametric regression}
    \label{fig:vm}
  \end{subfigure}
\end{figure}

As a baseline, we defined a \textit{na\"ive} model which performs a
hill-climbing search over the similarity function until it reaches a
(possibly local) maximum. Once a maximum as been found, the model
computes an estimate of $Z$ by linearly interpolating between sampled
rotations.

When $h_1$ is correct and as the true angle of rotation approaches 0,
the \textit{na\"ive} model should increase in both response and
rotation accuracy, because local maxima will be less likely (and
therefore, it is more likely that the global maxima will be
discovered). For larger angles of rotation, this model should have a
tendency to under-rotate, because it will find local maxima and end
prematurely. Its estimates of $Z$ will usually be overestimates (if it
finds the global maxima, and then terminates) or underestimates (the
rest of the time). This model should almost always be accurate when
$h_0$ is correct.

\subsection{Parametric (Von Mises) model}

Another strategy is to assume a parametric shape for $S$ and fit the
appropriate parameters. When $h_1$ is correct, it is likely that the
function will be approximately unimodal (shapes with rotational
symmetry would be multimodal). A reasonable assumption, then, is that
the likelihood follows a ``wrapped Gaussian'', or Von Mises,
distribution:
\begin{equation}
  S(X_b, X_R) \approx h\cdot{}p(R\ \vert\ \hat{\theta}, \kappa)=\frac{h}{2\pi X_0(\kappa)}e^{\kappa\cos(R-\hat{\theta})}
\end{equation}
where $\kappa$ is the concentration parameter, $\hat{\theta}$ is the
preferred direction, and $h$ is a scale parameter. We fit these
parameters by minimizing the mean squared error between this PDF and
the computed values of $S$. To choose the sequence of rotations, we
use the same hill-climbing strategy as in the \textit{na\"ive}
model.\TODO{a better strategy would have been to compute the variance
  in parameter estimates, or something...} Figure \ref{fig:vm} shows
an example of this parametric model, prior to estimating $Z$.

\subsection{Nonparametric (Bayesian Quadrature) model}

\begin{figure}[t]
  \centering
  \includegraphics[width=\textwidth]{../figures/bq_regression.pdf}
  \caption{Nonparametric model}
  \label{fig:bq}
\end{figure}

A more flexible strategy uses what is known as \emph{Bayesian
  quadrature} \cite{Diaconis:1988uo,OHagan:1991tx} to estimate $Z$.
Bayesian quadrature allows us to compute a posterior distrbution over
$Z$ by placing a Gaussian Process (GP) \cite{Rasmussen:2006vz} prior
on the log likelihood function $\log S$ and evaluating $\log S$ at a
particular set of points. For our purposes, we choose these points to
be a sequence of $N$ consecutive mental rotations be
$\mathbf{R}=\{R_1, \ldots{}, R_N\}$. To obtain a single estimate of
$Z$, we want compute the posterior mean:
\begin{equation}
  E[Z\ \vert \ \log S]=\int_{\log S}\left(\int_R \exp(\log{S(X_b,X_R)})p(R)\ \mathrm{d}R\right)\mathcal{N}\left(\log{S}\ \vert \ \mu_{\log S}, \Sigma_{\log S}\right)\ \mathrm{d}\log S
\end{equation}
As in \cite{Osborne:2012tm}, we can approximate this with:
\begin{equation}
  E[Z\ \vert \ \log S] \approx \int_R \mu_{S}(1 + \mu_\Delta) p(R)\ \mathrm{d}R
\end{equation} 
Where $\mu_x$ is the mean of a GP, as follows:

\begin{itemize}
\itemsep1pt\parskip0pt\parsep0pt
\item $\mu_\Delta$ is a regression over $\Delta=\mu_{\log S} - \log
  \mu_S$ given $\mathbf{R}_\Delta$, which consists of $\mathbf{R}$ and
  a set of intermediate \emph{candidate points}, as described in
  \cite{Osborne:2012tm}.
\item $\mu_S$ is the mean of a GP regression over $S$ given
  $\mathbf{R}$.
\item $\mu_{\log S}$ is the mean of a GP regression over $\log S$
  given $\mathbf{R}$.
\end{itemize}

\paragraph{Choosing the sequence of mental images}

Because we can only evaluate $S(X_a, X_R)$ at a small number $k$ of
rotations, we must make an explicit decision about which
$\mathbf{R}=\{R_1, \ldots{}, R_k\}$ to use. Ideally, we should choose
$\mathbf{R}$ such that it will give us the best approximation of
$p(X_a, X_b\vert h_1)$. One option is to choose each $R_i$ in a greedy
manner to minimize the variance of $Z$ (i.e.~give a more accurate
estimate of $S$).

\section{Results}

\begin{figure}[t]
  \centering
  \includegraphics[width=\textwidth]{../figures/model_rotations.pdf}
  \caption{Model rotations}
  \label{fig:rotations}
\end{figure}

\begin{figure}[t]
  \centering
  \includegraphics[width=\textwidth]{../figures/Z_accuracy.pdf}
  \caption{Accuracy in estimating $Z$.}
  \label{fig:accuracy}
\end{figure}

\section{Discussion}


% \renewcommand\bibsection{\subsubsection*{\refname}}
\renewcommand\refname{\normalsize{References}}
\bibliographystyle{ieeetr}
\bibliography{references}

\end{document}



