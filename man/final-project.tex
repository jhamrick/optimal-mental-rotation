\documentclass{article} % For LaTeX2e
\usepackage{nips12submit_e,times}
%\documentstyle[nips12submit_09,times,art10]{article} % For LaTeX 2.09

\usepackage{amsmath, amsthm, amssymb}
% \usepackage{natbib}
\usepackage{graphicx}
\usepackage{wrapfig}
\usepackage{subcaption}

\pagestyle{empty}

\title{Optimal strategies in mental rotation tasks}
\author{Jessica B.~Hamrick\\
  Department of Psychology\\
  University of California, Berkeley\\
  Berkeley, CA 94720\\
  \texttt{jhamrick@berkeley.edu}}

% The \author macro works with any number of authors. There are two commands
% used to separate the names and addresses of multiple authors: \And and \AND.
%
% Using \And between authors leaves it to \LaTeX{} to determine where to break
% the lines. Using \AND forces a linebreak at that point. So, if \LaTeX{}
% puts 3 of 4 authors names on the first line, and the last on the second
% line, try using \AND instead of \And before the third author name.

\newcommand{\TODO}[1]{\textcolor{red}{[TODO: #1]}}
\newcommand{\MSE}[0]{\mathrm{MSE}}
\newcommand{\ME}[0]{\mathrm{ME}}
\newcommand{\naive}[0]{na\"ive}
\newcommand{\Naive}[0]{Na\"ive}

\nipsfinalcopy % Uncomment for camera-ready version

\begin{document}

\maketitle

% \begin{abstract}
% \TODO{}
% \end{abstract}

\section{Introduction}

% 1. The big picture
% 2. Our contribution
% 3. How we do it
% 4. What's in the paper

Consider the objects in Figure \ref{fig:mental-rotation}. In each
panel, are the two depicted objects identical (except for a rotation),
or distinct? When presented with this mental rotation task, people
default to a strategy in which they visualize one object rotating
until it is congruent with the other \cite{Shepard1971}. There is
strong evidence for such \textit{mental imagery} or \textit{mental
  simulation}: we can imagine three-dimensional objects in our minds
and manipulate them, to a certain extent, as if they were real
\cite{Kosslyn:2009tj}.

However, the use of mental simulation is predicated on determining
appropriate parameters to give the simulation, and people's cognitive
constraints may furthermore place limitations on the duration or
precision of simulation. One hypothesis for how these issues are
handled argues that people use a ``rational'' solution, meaning that
it is optimal under the given constraints
\cite{Lieder:2012wg,Vul:2009wy,Griffiths2012a}.

In the case of the classic mental rotation task, we might ask: in what
direction should the object be rotated?  What are the requirements for
``congruency''? When should one stop rotating and accept the
hypothesis that the objects are different? In this project, I will
investigate the optimal computational solution to this task. Thus, the
specific question this project aims to answer is: what is the rational
computational solution to the mental rotation task, and does it
predict the people's behavior on the task?

\section{Background}

% We do not know the form of $S$, and so cannot compute this integral
% analytically. We cannot rely on a simple Monte Carlo simulation to
% estimate it, either: because mental rotation is costly in terms of
% cognitive resources and must be performed in a sequence, we cannot
% evaluate $S(I_b|I_M)$ at arbitrary $I_M$. Instead, we must choose a
% relatively small, sequential set of $I_M$ to estimate the
% integral. 

% Bayesian quadrature \cite{Diaconis:1988uo} provides an alternate
% method for evaluating the integral by placing a prior distribution on
% functions and then computing a posterior over values of the
% integral. While Bayesian quadrature itself does not address the issue
% of choosing points to evaluate the function at, recent work in
% statistics and machine learning has examined how to optimally select
% these samples \cite{Osborne:2012tm}.


\section{Computational model}

People are presented with two images, $X_a$ and $X_b$, which are the
coordinates of the vertices of two shapes (e.g., see Figure
\ref{fig:stimuli}). Participants must determine whether $X_a$ and
$X_b$ were generated from the same (albeit possibly transformed and
permuted) original shape, i.e., whether $\exists R,M\textrm{ s.t. }
X_b=MRX_a$, where $M$ is a permutation matrix and $R$ is a rotation
matrix.

We can formulate the judgment of whether $X_a$ and $X_b$ have the same
origins by deciding about two hypotheses:

\begin{itemize}
\itemsep1pt\parskip0pt\parsep0pt
\item
  $h_0$: $\forall M,R\ X_b\neq MRX_a$
\item
  $h_1$: $\exists M,R\textrm{ s.t. } X_b=MRX_a$
\end{itemize}

To compare the hypotheses, we need to compute the likelihood of each
hypothesis, i.e. $p(X_a,\ X_b\ \vert \ h_0)$ and $p(X_a,\ X_b\ \vert \
h_1)$.

The likelihood is easy to compute under $h_0$, because we assume that
$X_a$ and $X_b$ are independent. Thus:
\begin{equation}
  p(X_a,\ X_b\ \vert \ h_0)=p(X_a)p(X_b)
  \label{eq:lh-h0}
\end{equation}

When the configurations are the same, the likelihood becomes more
complicated:
\begin{equation} 
  p(X_a,\ X_b\ \vert \ h_1)=\int_R\int_M p(X_a) p(X_b\vert X_a,R,M) p(R) p(M)\ \mathrm{d}M\ \mathrm{d}R
\end{equation}

For a small number of vertices, we can compute the integral over $M$
by enumerating every possible mapping between $X_a$ and $X_b$. After
doing so, we obtain:
\begin{equation} 
  p(X_a,\ X_b\ \vert \ h_1)=\int_R p(X_a) p(X_b\vert X_a,R) p(R)\ \mathrm{d}R
\end{equation}

\begin{figure}[t]
  \centering
  \includegraphics[width=0.95\textwidth]{../figures/shepard-rotation.png}
  \caption{\textbf{Classic mental rotation task}. Participants in
    \cite{Shepard1971} saw stimuli such as these, and judged whether
    each pair of shapes was the same shape in two different
    orientations, or two different shapes. \textbf{A} shows a ``plane
    rotation'', \textbf{B} shows a ``depth rotation'', and \textbf{C}
    shows two distinct objects.}
  \label{fig:mental-rotation}
\end{figure}

However, we cannot compute $p(X_b\vert X_a, R)$ directly. Instead, we
introduce a new variable $X_R$ denoting a mental image, which
approximates $RX_a$. The $X_R$ are generated sequentially by repeated
application of a function $\tau$:
\begin{align}
  X_R&=RX_a\nonumber \\
  &=\tau(X_{R-r}, r)\nonumber \\
  &=\tau(\tau(X_{R-2r}, r), r)\nonumber \\
  &\ldots{}\nonumber \\
  &=\tau^{(\frac{R}{r})}(X_a, r)
  \label{eq:tau}
\end{align} 
Where $r$ is a small angle, and $\tau^{(i)}$ indicates $i$ recursive
applications of $\tau$. Using this sequential function, we get:
\begin{align}
  p(X_a, X_b\ \vert \ h_1)&=\int_R \int_{X} p(X_b\vert X) p(X\vert X_a, R)p(X_a)p(R)\ \mathrm{d}X\ \mathrm{d}R \nonumber \\
  x&= \int_R \int_X p(X_b\vert X)\delta(\tau^{(\frac{R}{r})}(X_a, r)-X)p(X_a)p(R)\ \mathrm{d}X\ \mathrm{d}R \nonumber \\
  &= \int_R p(X_b\vert X_R)p(X_a)p(R)\ \mathrm{d}R
  \label{eq:lh-h1}
\end{align}

Once we have computed both likelihoods, we compute their ratio:
\begin{equation}
  \ell=\frac{p(X_a, X_b\ \vert \ h_1)}{p(X_a, X_b\ \vert \ h_0)}=\frac{\int_R p(X_b\ \vert\ X_R)p(R)\ \mathrm{d}R}{p(X_b)}
  \label{eq:lh-ratio}
\end{equation}
If $\ell<1$, then $h_0$ is the more likely hypothesis. If $\ell>1$,
then $h_1$ is the more likely hypothesis.

\section{Implementation}


\begin{figure}[t]
  \centering
  \begin{subfigure}[b]{0.45\textwidth}
    \centering
    \includegraphics[width=\textwidth]{../figures/stimuli_shapes.pdf}
    \vspace{0pt}
    \caption{Example stimuli}
    \label{fig:stimuli}
  \end{subfigure}
  \begin{subfigure}[b]{0.45\textwidth}
    \centering
    \includegraphics[width=\textwidth]{../figures/likelihood_function.pdf}
    \caption{Likelihood function}
    \label{fig:likelihood}
  \end{subfigure}
  \caption{\textbf{Rotated shapes and their similarity.}  \textbf{(a)}
    An example stimulus in which the shapes differ only by a
    rotation. All stimuli consist of five vertices centered around the
    origin, and four edges which create a closed loop from the
    vertices. \textbf{(b)} The approximate likelihood (similarity) of
    $X_b$ given $X_R$, where $X_R$ is a rotation of $X_a$ by the angle
    $R$. The true angle of rotation between $X_a$ and $X_b$ is
    slightly less than $\pi$, corresponding to the global maximum in
    the likelihood function.}
\end{figure}

We define the prior probabilities over stimuli based on how they are
generated. For shapes with $n$ vertices, each vertex is at a random
angle with a radius random chosen between 0 and 1, in polar
coordinates, and could be chosen in any of $n!$ different ways. Thus,
the prior over a shape $X$ is:
\begin{equation}
  p(X)=n!\left(\frac{1}{2\pi}\right)^n
  \label{eq:prior}
\end{equation} 
which gives us the denominator in Equation \ref{eq:lh-ratio}. 

Computing the numerator of Equation \ref{eq:lh-ratio} is more
difficult, as we do know $p(X_b\vert X_R)$. We approximate it with a
similarity function $S(X_b, X_R)$, which also takes into account the
different possible mappings of vertices:
\begin{equation}
  Z=\int_R S(X_b, X_R)p(R)\ \mathrm{d}R\ \approx\int_Rp(X_b\ \vert\ X_R)p(R)\ \mathrm{d}R
  \label{eq:Z}
\end{equation}

Because the vertices are connected in a way which forms a closed loop,
we need only consider $n$ mappings, $M$, of the $n$ vertices (we
assume the uncertainty is only in which is the ``first''
vertex). Thus, the possible orderings are of the form $M=\lbrace{}0,
1, \ldots{}, n\rbrace{}$, $M=\lbrace{}n, 0, \ldots{}, n-1\rbrace{}$,
and so on. This gives us an explicit form for the similarity function:
\begin{equation}
  S(X_b, X_R)=\frac{1}{n}\sum_{M\in\mathbb{M}}\prod_{i=1}^n\mathcal{N}(X_b[i]\ \vert \ (MX_R)[i], \Sigma)
  \label{eq:similarity}
\end{equation}
where $i$ denotes the $i^{th}$ vertex. An example of $S$ for the
stimuli shown in Figure \ref{fig:stimuli} is illustrated in Figure
\ref{fig:likelihood}.

This process of mental rotation (i.e., generating a single $X_R$ (as
in Equation \ref{eq:tau}) and then computing $S(X_b, X_R)$) is a
computationally demanding cognitive process. Thus, our goal is to
minimize the number of rotations while still obtaining an estimate of
$Z$ that is accurate enough to choose the correct hypothesis. With
this in mind, we now examine several approaches to estimating $Z$.

In each of these models, we denote the computed rotations to be a set
$\mathbf{R}=\{R_1, R_2, \ldots{}\}$.

\subsection{Gold standard}

To compare the accuracy of other models' estimates of $Z$, we computed
a ``gold standard''\footnote{This only gives an accurate estimate of
  $Z$, which is itself an approximation, and is thus not necessarily
  the true value of the numerator in Equation \ref{eq:lh-ratio}.} by
evaluating $S(X_b, X_R)$ at 360 values of $R$ spaced evenly between
$0$ and $2\pi$.

\subsection{\Naive{} model}

\begin{figure}[t]
  \centering
  \begin{subfigure}[b]{0.45\textwidth}
    \centering
    \includegraphics[width=\textwidth]{../figures/li_regression.pdf}
    \caption{\Naive{} model}
    \label{fig:li}
  \end{subfigure}
  \begin{subfigure}[b]{0.45\textwidth}
    \centering
    \includegraphics[width=\textwidth]{../figures/vm_regression.pdf}
    \caption{Parametric model}
    \label{fig:vm}
  \end{subfigure}
  \caption{\textbf{Simple methods of estimating $S$.} In both cases,
    hill-climbing search is used until a maxima is found (in this
    case, at approximately $\frac{\pi}{6}$). The sampled points
    $\mathbf{R}$ (red circles) are then used to estimate $S$ (black
    lines are the true $S$, red lines are the estimate). \textbf{(a)}
    Linear interpolation. The overall estimate of $Z$ here will be too
    low, but certain angles will have a disproportionate contribution
    (e.g., between $\frac{\pi}{4}$ and $\frac{3\pi}{4}$. \textbf{(b)}
    Best fit of scaled Von Mises PDF parameters. As in (a), $Z$ will
    be underestimated. However, the fit around the maximum near
    $\frac{\pi}{6}$ is much more accurate.}
  \label{fig:simple-models}
\end{figure}

As a baseline, we defined a \naive{} model which performs a
hill-climbing search over the similarity function until it reaches a
(possibly local) maximum. Once a maximum as been found, the model
computes an estimate of $Z$ by linearly interpolating between sampled
rotations. Figure \ref{fig:li} shows an example of the \naive{} model
prior to estimating $Z$.

\subsection{Parametric (Von Mises) model}

Another strategy is to assume a parametric shape for $S$ and fit the
appropriate parameters. When $h_1$ is correct, it is likely that the
function will be approximately unimodal (shapes with rotational
symmetry would be multimodal). A reasonable assumption, then, is that
the likelihood follows a ``wrapped Gaussian'', or Von Mises,
distribution:
\begin{equation}
  S(X_b, X_R) \approx h\cdot{}p(R\ \vert\ \hat{\theta}, \kappa)=\frac{h}{2\pi I_0(\kappa)}e^{\kappa\cos(R-\hat{\theta})}
\end{equation}
where $\kappa$ is the concentration parameter, $\hat{\theta}$ is the
preferred direction, $h$ is a scale parameter, and $I_0$ is the
modified Bessel function of order zero. We fit these parameters by
minimizing the mean squared error between this PDF and the computed
values of $S$. To choose the sequence of rotations, we use the same
hill-climbing strategy as in the \naive{} model. Figure \ref{fig:vm}
shows an example of this parametric model, prior to estimating $Z$.

\subsection{Nonparametric (Bayesian Quadrature) model}

\begin{figure}[t]
  \centering
  \includegraphics[width=\textwidth]{../figures/bq_regression.pdf}
  \caption{\textbf{Nonparametric model.} Each panel shows one step of
    the Bayesian Quadrature regression. Upper left: the original
    Gaussian Process (GP) regression for $S$. Lower left: the GP
    regression for $\log(S+1)$. Lower right: the GP regression for
    $\Delta_c=\mu_{\log S} - \log \mu_S$. Upper right: the adjusted
    regression for $S$, where the mean is equal to
    $\mu_S(1+\mu_{\Delta_c})$. The model uses this final estimate to
    compute $Z$ and will continue rotating until the variance of $Z$
    is low enough that a hypothesis may be accepted. This method
    allows the model to avoid local maxima such as the one near
    $\frac{\pi}{6}$, which causes trouble for the \naive{} and
    parametric models in Figure \ref{fig:simple-models}.}
  \label{fig:bq}
\end{figure}

A more flexible strategy uses what is known as \emph{Bayesian
  Quadrature} \cite{Diaconis:1988uo,OHagan:1991tx} to estimate $Z$.
Bayesian Quadrature allows us to compute a posterior distrbution over
$Z$ by placing a Gaussian Process (GP) prior on the function $S$ and
evaluating $S$ at a particular set of points. Because our data is
circular, we can use a periodic kernel \cite{Rasmussen:2006vz}:
\begin{equation}
k(R, R^\prime)=h^2\exp\left(-\frac{2\sin^2\left(\frac{1}{2}(R-R^\prime)\right)}{w^2}\right)
\end{equation}

Bayesian Quadrature has its difficulties, however. While in our case
$S$ is a non-negative likelihood function, GP regression enforces no
such constraint. In an effort to avoid this problem,
\cite{Osborne:2012tm} give a method which involves instead placing a
prior over the log likelihood\footnote{In practice, as in
  \cite{Osborne:2012tm}, we use the transform of
  $\log(S+1)$. Additionally, while \cite{Osborne:2012tm} use a
  combination of MLII and marginalization to fit the kernel
  parameters, we set the output scale $h_S=0.125$ and $h_{\log
    S}=\log(h_S + 1)$, and use MLII to fit all other parameters.},
thus ensuring that $S=e^{\log S}$ will be positive\footnote{We are not
  guaranteed positivity, however, the approximation to the integral
  over $\log S$ (Equation \ref{eq:bq-Z-mean}) requires computing
  $\mu_S$, which may have nonpositive segments.}:
\begin{equation*}
  E[Z\ \vert \ \log S]=\int_{\log S}\left(\int_R \exp(\log{S(X_b,X_R)})p(R)\ \mathrm{d}R\right)\mathcal{N}\left(\log{S}\ \vert \ \mu_{\log S}, \Sigma_{\log S}\right)\ \mathrm{d}\log S
\end{equation*}
where $\mu_{\log S}$ and $\Sigma_{\log S}$ are the mean and
covariance, respectively, of the GP regression over $\log S$ given
$\mathbf{R}$. We approximate this according to the method given in
\cite{Osborne:2012tm}:
\begin{equation}
  \mu_Z=E[Z\ \vert \ S, \log S, \Delta_c] \approx \int_R \mu_{S}(1 + \mu_{\Delta_c}) p(R)\ \mathrm{d}R 
  \label{eq:bq-Z-mean}
\end{equation}
where $\mu_S$ is the mean of a GP regression over $S$ given
$\mathbf{R}$; and $\mu_{\Delta_c}$ is a regression over
$\Delta_c=\mu_{\log S} - \log \mu_S$ given $\mathbf{R}_c$, which
consists of $\mathbf{R}$ and a set of intermediate \emph{candidate
  points}\footnote{These candidate points do not require evaluating
  the true $S$, only the GP estimates of $S$ and $\log S$.} $c$ as
described in \cite{Osborne:2012tm}. The variance of the estimate of
$Z$ is given by:
\begin{equation}
  \sigma_Z=\mathrm{Var}(Z\ \vert \ S, \log S, \Delta_c) = \int_R\int_{R^\prime} \mu_S(R)\mu_S(R^\prime) \Sigma_{\log S}(R, R^\prime)p(R)p(R^\prime)\ \mathrm{d}R\ \mathrm{d}R^\prime
  \label{eq:bq-Z-var}
\end{equation}

To choose $\mathbf{R}$, we pick an initial direction of rotation which
results in the higher value of $S$, and then continue rotating in that
direction until the variance in Equation \ref{eq:bq-Z-var} is low
enough that we are confident about the likelihood ratio $\ell$:
\begin{equation*}
p(\ell)\approx\frac{1}{p(X_b)}\ \mathcal{N}(Z\ \vert\ \mu_Z, \sigma_z)
\end{equation*}
Specifically, we choose $h_0$ when $p(\ell < 1)\geq 0.95$, and we
choose $h_1$ when $p(\ell > 1)\geq 0.95$. Until one of these
conditions are met (or the shape has been fully rotated), the model
will continue to compute rotations and update its estimate of $Z$.

\section{Results}

\begin{figure}[t]
  \centering
  \includegraphics[width=\textwidth]{../figures/model_rotations.pdf}
  \caption{\textbf{Model rotations.} Each subplot shows the
    correspondence between the true angle of rotation ($R$) for $h_1$
    stimuli and the amount of rotation performed by the model. Dotted
    lines indicate a perfect 1:1 relationship. The \naive{} and
    parametric models (left and center panels, respectively) are
    identical; they show modest agreement until rougly
    $\frac{\pi}{2}$, after which agreement decreases as they tend to
    encounter local maxima. The nonparametric model (right panel)
    exhibits a significant correlation between the true angle and the
    amount the model rotates.}
  \label{fig:rotations}
\end{figure}

We evaluated each model's performance on 50 randomly generated shapes
(such as $X_a$ in Figure \ref{fig:stimuli}). For each shape, we picked
a random angle $R$ and generated two stimuli: a $h_0$ stimulus, by
reflecting $X_a$ across the $y$-axis and rotating it by $R$; and a
$h_1$ stimulus, by simply rotating $X_a$ by $R$.

We considered three metrics of performance in particular:
\begin{itemize}
\item \textit{Rotations}: for stimuli where $h_1$ is correct, how
  correlated are the true angles of rotation with the rotations
  performed by the model? This is defined to be the Pearson's
  correlation coefficient $\rho$ between $R$ and $\vert
  \mathbf{R}\vert$. Figure \ref{fig:rotations} shows individual points
  corresponding to true rotation vs. rotation by the model, for each
  model.
\item \textit{Responses}: how accurate is the model at choosing the
  correct hypothesis? This is defined to be the mean error ($\ME{}$),
  or fraction of times the model picks the incorrect hypothesis.
\item \textit{Estimates of $Z$}: how accurate is the model's estimate
  of $Z$? This is defined to be the mean squared error ($\MSE{}$)
  between the model's estimate of $Z$ and the ``gold standard'' value,
  where the error has scale such that $\MSE{}=0$ indicates no error,
  and $\MSE{}=1$ indicates maximum error. Figure \ref{fig:accuracy}
  shows plots of the true (gold standard) value of $Z$ vs. the model's
  estimate, for each model.
\end{itemize}

\paragraph{Gold standard} 

The gold standard always evaluates $S$ at all angles, so the
correlation between it and the true angle of rotation is
undefined. The mean error of responses was $\ME{}=0.01$, indicating
that $Z$ is a sufficient approximation of $\int_R p(X_b\ \vert\
X_R)p(R)\ \mathrm{d}R$.

\paragraph{\Naive{} model} 

The correlation between the number of rotations computed by the
\naive{} model and the true angle of rotation was $\rho=-0.11$. The
shape of the data is more interesting (see Figure \ref{fig:rotations},
left panel): the \naive{} model actually corresponds quite well to
the true angle of rotation for $R<\frac{\pi}{2}$ ($\rho=0.64$). This
is unsurprising, because the closer the true angle is to zero, the
less the model has to rotate, and the less likely it will get stuck on
local maxima. Thus, it is more likely to locat the global maximum,
which generally corresponds to the true angle of rotation. For
$R>\frac{\pi}{2}$, we see an increasing tendency to under-rotate
($\rho=-0.42$); this is likely because it finds a local maxima and
ends prematurely.

The \naive{} model's response error rate was better than chance, but
still quite high, with $\ME{}=0.23$. Closer inspection reveals that
the bulk of this comes from $h_1$ stimuli ($\ME{}=0.36$
vs. $\ME{}=0.1$ for $h_0$ stimuli). This is, as above, probably
related to finding only local maxima: if the model finds a local
maxima which is low enough, the area under the estimated curve will
not be large enough to accept $h_1$.

This is reflected in the accuracy in estimating $Z$ as well
($\MSE{}=0.14$). Figure \ref{fig:accuracy} (left panel) shows
individual estimates for each stimulus, color-coded by true
hypothesis. The model underestimates $Z$ for significant portion of
the $h_1$ stimuli. On the other $h_1$ stimuli, the \naive{} model
actually tends to overestimate because linear interpolation does not
account for symmetry of global maxima (e.g., Figure \ref{fig:li}).

\paragraph{Parametric (Von Mises) model}

Like the \naive{} model, the parametric model uses hill-climbing as a
sampling strategy, so they have identical correlations with the true
angle of rotation. However, given the same $\mathbf{R}$, the
parametric model estimates $Z$ differently, thus giving rise to
different response and estimated $Z$ error rates. The response error
for the parametric model was higher than the \naive{} model, with
$\ME{}=0.26$. Breaking this down, we see that the error for $h_1$
stimuli is $\ME{}=0.5$: in other words, the parametric model is at
chance when determining whether two identical shapes are the
same. This is because it never overestimates (see Figure
\ref{fig:accuracy}, center panel). So, it accurately estimates $Z$ for
stimuli which the \naive{} model would overestimate ($\MSE{}=0.07$),
but still has poor performance when $Z$ is underestimated. Indeed, if
we exclude stimuli for which the \naive{} model overestimates ($Z\geq
0.3$), the error for the \naive{} model lowers to match that of the
parametric model.

\paragraph{Nonparametric (Bayesian Quadrature) model}

The rotations computed by the nonparametric model were strongly
correlated with the true rotations ($\rho=0.73$, see Figure
\ref{fig:rotations}, right panel). This is because the nonparametric
model does not get stuck as easily on local optima: it will continue
rotating until it is confident that its estimate of $Z$ is accurate. 

Correspondingly, the nonparametric model is much more accurate in
choosing the correct hypothesis ($\ME{}=0.06$). As with the \naive{}
and parametric models, it mostly errs on $h_1$ stimuli
($\ME{}=0.1$). This appears to mostly be the result of choosing the
incorrect initial direction of rotation.

Similarly, because the nonparametric model actively attemps to obtain
an accurate estimate of $Z$, its estimate of $Z$ is indeed fairly
accurate ($\MSE{}=0.10$). This is slightly higher than the
nonparametric model, but is due to a few outliers. If we again exclude
stimuli for which $Z$ is greatly overestimated ($Z\geq 0.3$), the
nonparametric model's error decreases significantly ($\MSE{}=0.03$).

\begin{figure}[t]
  \centering
  \includegraphics[width=\textwidth]{../figures/Z_accuracy.pdf}
  \caption{\textbf{Accuracy in estimating $Z$.} Each subplot shows the
    true (``gold standard'') value of $Z$ vs. the value estimated by
    the model. Black dotted lines indicate a perfect 1:1
    correspondence between the true and estimated values. All models
    perform well for $h_0$ stimuli, but exhibit different behavior for
    $h_1$ stimuli. The \naive{} model (left panel) tends to either
    greatly overestimate, or underestimate. The parametric model
    (center panel) is very accurate for about half the $h_1$ stimuli,
    and severly underestimates the rest. The nonparametric model
    (right panel) maintains a decent correspondence with the true $Z$,
    with the exception of a handful of outliers.}
  \label{fig:accuracy}
\end{figure}

\section{Discussion}

We investigated strategies for performing the mental rotation task
\cite{Shepard1971} in two dimensions, and found that a nonparametric
model that enforces a positivity constraint on the likelihood function
performs best, as it is able to actively monitor the confidence of its
estimate. The simpler models performed much worse: they did not
maintain the linear relationship with the true angle of rotation due
to lack of robustness against local maxima, and they were inaccurate
at identifying whether the two shapes were the same or not.

One option for improving the performance of the simple models would be
to fit a threshold value, below which local maxima would be
ignored. However, this strategy would be rather brittle, for if the
distribution of shapes changed, the threshold would have to be
re-learned. It is possible that people exhibit this behavior, but we
cannot make any assumptions one way or the other without empirical
data.

A further strength of the nonparametric model over the other models,
however, is that it is likely to generalize well to three
dimensions. One aspect of \cite{Osborne:2012tm} which we have not yet
explored is their main contribution of \textit{active sampling}, in
which samples are iteratively selected to maximally decrease the
expected variance of $Z$. Due to the sequential constraint of mental
rotation, this strategy is not particularly useful in two dimensions,
as there are only every two directions in which to rotate. In three
dimensions, however, there are an infinite number of directions that
could be chosen after every step.

We conclude that, from this initial survey, models of mental rotation
which take an ``active'', directed approach seem well-suited to
explaining human behavior in these tasks. Future work will collect
empirical data from participants to perform a more quantitative
analysis of these modeling tools.

% \textbf{Acknowledgments}
% \TODO{}


% \renewcommand\bibsection{\subsubsection*{\refname}}
\renewcommand\refname{\normalsize{References}}
\bibliographystyle{ieeetr}
\bibliography{references}

\end{document}



