% 
% Annual Cognitive Science Conference
% Sample LaTeX Paper -- Proceedings Format
% 

% Original : Ashwin Ram (ashwin@cc.gatech.edu)       04/01/1994
% Modified : Johanna Moore (jmoore@cs.pitt.edu)      03/17/1995
% Modified : David Noelle (noelle@ucsd.edu)          03/15/1996
% Modified : Pat Langley (langley@cs.stanford.edu)   01/26/1997
% Latex2e corrections by Ramin Charles Nakisa        01/28/1997 
% Modified : Tina Eliassi-Rad (eliassi@cs.wisc.edu)  01/31/1998
% Modified : Trisha Yannuzzi (trisha@ircs.upenn.edu) 12/28/1999 (in process)
% Modified : Mary Ellen Foster (M.E.Foster@ed.ac.uk) 12/11/2000
% Modified : Ken Forbus                              01/23/2004
% Modified : Eli M. Silk (esilk@pitt.edu)            05/24/2005
% Modified: Niels Taatgen (taatgen@cmu.edu) 10/24/2006

%% Change ``a4paper'' in the following line to ``letterpaper'' if you are
%% producing a letter-format document.

\documentclass[10pt,letterpaper]{article}

\usepackage{cogsci}
\usepackage{pslatex}
\usepackage{apacite}
\usepackage{graphicx}
\usepackage{color}
\usepackage{amsmath, amsthm, amssymb}

\newcommand{\TODO}[1]{\textcolor{red}{[TODO: #1]}}
\newcommand{\Xa}[0]{\mathbf{X}_a}
\newcommand{\Xb}[0]{\mathbf{X}_b}
\newcommand{\Xt}[0]{\mathbf{X}_t}
\newcommand{\R}[0]{\mathbf{R}_\theta}
\newcommand{\F}[0]{\mathbf{F}}
\newcommand{\M}[0]{\mathbf{M}}
\newcommand{\I}[0]{\mathbb{I}}
\newcommand{\hi}[0]{h=0}
\newcommand{\hf}[0]{h=1}

\newenvironment{pitemize}{
\begin{itemize}
  \setlength{\itemsep}{1pt}
  \setlength{\parskip}{0pt}
  \setlength{\parsep}{0pt}
}{\end{itemize}}


\title{An active sampling model of mental rotation}
 
\author{{\large \bf Jessica B. Hamrick (jhamrick@berkeley.edu)} \\
  {\large \bf Thomas L. Griffiths (tom\_griffiths@berkeley.edu)} \\
  Department of Psychology, University of California, Berkeley, CA
  94720 USA}

\begin{document}

\maketitle


\begin{abstract}
\TODO{}

\textbf{Keywords:} 
\TODO{}
\end{abstract}


\section{Introduction}

One of our most astonishing cognitive feats is the ability to
envision, manipulate, and plan with objects -- all without actually
perceiving them. This capacity for ``mental simulation'' has been
widely studied and has sparked intense debate about the underlying
representation of mental images \TODO{cite}. Despite this vast
literature, there have been surprisingly few computational models of
mental imagery.



\section{Background}

classic result by shepard and metzler

assume people rotate until congruent

but how do people actually construct the plan for the rotation?

evidence that people *don't* actually rotate until congruent

so what are they actually doing?

perhaps they are performing mental manipulations of an image, but not
a single trajectory of rotation

rational analysis \cite{Marr:1983to,anderson90,Shepard:1987tt}

1) the problem: determine something about the spatial transformations
that an object has undergone
2) the goal: make this determiniation while using the least amount of
resources
3) the solution: perform those rotations and reflections which will
give the most information (?) about the answer until an answer is
found

motivate the active searching approach
\cite{Gureckis:2012gu,Markant:2012uu} \cite{Markant:2012uu,Nelson2007}

we focus less on the representation, and more on the process, assuming
a spatial representation (because you need both to make a claim,
\cite{Anderson1978})

problem solving \cite{Hegarty2004, Schwartz1999}

symbolic vs simulation \cite{Schwartz:1996uy}

eye tracking \cite{Just1976}

\section{Models of mental rotation}

In the mental rotation task, people are presented with pairs of
two-dimensional shapes similar to those used by \cite{Cooper:1975wp}
(e.g., Figure \ref{fig:stimuli}) and must determine whether the two
images depict the same shape, or mirror-image shapes.

\subsection{Computational-level analysis}

Formally, we denote the shapes as $X_a$ and $X_b$ and assume $X_b$ is
generated by a transformation of $X_a$, i.e. $X_b=f(X_a, \theta, h)$,
where $\theta$ is a rotation, $h=0$ is the hypothesis that the images
depict the same object, and $h=1$ is the hypothesis that the images
depict mirror-image objects. The posterior probability of each
hypothesis given the observed shapes is then:
\begin{equation}
  p(h\ \vert\ X_a, X_b) \propto \int p(X_b\ \vert\ X_a, \theta, h)p(X_a)p(h)p(\theta)\ \mathrm{d}\theta.
  \label{eq:posterior}
\end{equation}
Because we ultimately want to determine which hypothesis is more
likely, the quantity of interest is a ratio:
\begin{equation*}
  \ell := \frac{\int p(X_b\ \vert\ X_a, \theta, \hi)p(X_a)p(\hi)p(\theta)\ \mathrm{d}\theta}{\int p(X_b\ \vert\ X_a, \theta, \hf)p(X_a)p(\hf)p(\theta)\ \mathrm{d}\theta},
\end{equation*}
which (assuming the two hypotheses are equally likely \textit{a
  priori}, and that all rotations are equally likely) simplifies to:
\begin{equation}
  \ell = \frac{\int p(X_b\ \vert\ X_a, \theta, \hi)\ \mathrm{d}\theta}{\int p(X_b\ \vert\ X_a, \theta, \hf)\ \mathrm{d}\theta}.
  \label{eq:lh-ratio}
\end{equation}
If $\ell > 1$, then we accept the hypothesis that the images depict
the same object ($\hi$); if $\ell < 1$, then we accept the hypothesis
that the images depict flipped objects ($\hf$).

\subsection{Algorithmic assumptions}

If we represent a shape of $N$ vertices with a $N\times 2$ coordinate
matrix $\mathbf{X}=[\mathbf{x}_1, \ldots{}, \mathbf{x}_N]$, then the
transformation $f$ is $f(\mathbf{X}, h,
\theta):=\mathbf{X}\F_h^T\R^T$, where $\R$ is a rotation matrix,
$\F_0$ is the identity matrix $\I$, and $\F_1$ is a reflection matrix
across the $y$-axis. Assuming no computational constraints, the
simplest solution is to compute the left inverse of $\Xa\F_h^T$ and,
for each $h$, check whether $(\Xa
\F_h^T)_\mathrm{left}^{-1}\cdot{}\Xb$ is a valid rotation matrix.

However, we assume that the ability to compute left inverses is not
available. Instead, the observed shapes may only be transformed by a
small amount at a time, and each transformation is costly in terms of
computational resources. The goal, then is to estimate the integrals
in Eq.~\ref{eq:lh-ratio} by evaluating $p(\Xb\ \vert\ \Xa, \theta, h)$
as few times as possible. 

We define $p(\Xb\ \vert\ \Xa, \theta, h)$ to be the similarity between
$\Xb$ and a transformation of $\Xa$:
\begin{equation}
  p(\Xb\ \vert\ \Xa, \theta, h):= S(\Xb, f(\Xa, h, \theta)).
  \label{eq:likelihood}
\end{equation}
We do not know which vertices of $\Xb$ correspond to which vertices of
$\Xa$, so $S$ must marginalize over the set of possible mappings,
$\mathbb{M}$. For brevity, let $\mathbf{X}_m=\M\cdot{}f(\Xa, h,
\theta)$ where $\M\in\mathbb{M}$ is a permutation matrix. Then:
\begin{equation}
  S(\Xb, f(\Xa, h, \theta)):=\frac{1}{|\mathbb{M}|} \sum_{\M} \prod_{n=1}^N \mathcal{N}(\mathbf{x}_{bn}\ \vert \ \mathbf{x}_{mn}, \I\sigma^2)
  \label{eq:similarity}
\end{equation}

Additionally, we have the constraint of small
transformations. Specifically, if the current mental image is $\Xt$,
then:
\begin{equation}
  \mathbf{X}_{t+1} = \left\{ \begin{array}{ll}
      f(\Xt, 0, \epsilon) &\mbox{ rotate right by $\epsilon$,} \\
      f(\Xt, 0, -\epsilon) &\mbox{ rotate left by $\epsilon$,} \\
      f(\Xt, 1, 0) &\mbox{ flip,} \\
      f(\Xa, 0, 0) &\mbox{ reset, or} \\
      f(\Xa, 1, 0) &\mbox{ reset and flip.} \\
    \end{array} \right.
\end{equation}

\subsubsection{Hill climbing model}

An intuitive and simple model is to first rotate in the direction that
increases similarity, and then stop rotating once a ``match'' is
found. If all rotations have been exhausted, then flip, and try
rotating again.


\subsubsection{Bayesian quadrature model}


\section{Methods}


\subsection{Stimuli}


\subsection{Procedure}


\section{Results}


\section{Discussion}


\section{Acknowledgments}

This research was supported by ONR MURI grant number N00014-13-1-0341,
and a Berkeley Fellowship awarded to JBH.

\bibliographystyle{apacite}

\setlength{\bibleftmargin}{.125in}
\setlength{\bibindent}{-\bibleftmargin}

\bibliography{references}

\end{document}
