 
% Annual Cognitive Science Conference
% Sample LaTeX Paper -- Proceedings Format
% 

% Original : Ashwin Ram (ashwin@cc.gatech.edu)       04/01/1994
% Modified : Johanna Moore (jmoore@cs.pitt.edu)      03/17/1995
% Modified : David Noelle (noelle@ucsd.edu)          03/15/1996
% Modified : Pat Langley (langley@cs.stanford.edu)   01/26/1997
% Latex2e corrections by Ramin Charles Nakisa        01/28/1997 
% Modified : Tina Eliassi-Rad (eliassi@cs.wisc.edu)  01/31/1998
% Modified : Trisha Yannuzzi (trisha@ircs.upenn.edu) 12/28/1999 (in process)
% Modified : Mary Ellen Foster (M.E.Foster@ed.ac.uk) 12/11/2000
% Modified : Ken Forbus                              01/23/2004
% Modified : Eli M. Silk (esilk@pitt.edu)            05/24/2005
% Modified: Niels Taatgen (taatgen@cmu.edu) 10/24/2006

%% Change ``a4paper'' in the following line to ``letterpaper'' if you are
%% producing a letter-format document.

\documentclass[10pt,letterpaper]{article}

%% For better looking math (but it takes up more space):
% \AtBeginDocument{\RequirePackage{lmodern, times}}

\usepackage{cogsci}
\usepackage{pslatex}
\usepackage{apacite}
\usepackage{graphicx}
\usepackage{color}
\usepackage{amsmath, amsthm, amssymb}
\usepackage{subcaption}
\usepackage{caption}

\newcommand{\TODO}[1]{\textcolor{red}{[TODO: #1]}}
\newcommand{\Xa}[0]{\mathbf{X}_a}
\newcommand{\Xb}[0]{\mathbf{X}_b}
\newcommand{\Xt}[0]{\mathbf{X}_t}
\newcommand{\R}[0]{\mathbf{R}_\theta}
\newcommand{\F}[0]{\mathbf{F}}
\newcommand{\M}[0]{\mathbf{M}}
\newcommand{\I}[0]{\mathbb{I}}
\newcommand{\hi}[0]{h=0}
\newcommand{\hf}[0]{h=1}
\newcommand{\dif}[0]{\,\mathrm{d}}

\newcommand{\Oc}[0]{Oracle}
\newcommand{\Th}[0]{Threshold}
\newcommand{\Hc}[0]{HC}
\newcommand{\Bq}[0]{BQ}

\newenvironment{pitemize}{
\begin{itemize}
  \setlength{\itemsep}{1pt}
  \setlength{\parskip}{0pt}
  \setlength{\parsep}{0pt}
}{\end{itemize}}

% space below "Figure 1: ...", but only for inline figures
%\addtolength{\textfloatsep}{-0.5cm}

\addtolength{\abovecaptionskip}{-0.25cm}
\addtolength{\belowcaptionskip}{-0.3cm}

%% AUTOMATICALLY GENERATED -- DO NOT EDIT!
\newcommand{\BqTime}[0]{$M=20.5$, 95\% CI $[20.4, 20.6]$}
\newcommand{\BqpTime}[0]{$M=20.5$, 95\% CI $[20.4, 20.6]$}
\newcommand{\ExpTime}[0]{$M=1981.1$, 95\% CI $[1969.4, 1992.1]$}
\newcommand{\ExpaTime}[0]{$M=2046.5$, 95\% CI $[2028.6, 2065.3]$}
\newcommand{\ExpbTime}[0]{$M=1921.7$, 95\% CI $[1905.3, 1936.0]$}
\newcommand{\GsTime}[0]{$M=722.0$, 95\% CI $[722.0, 722.0]$}
\newcommand{\HcTime}[0]{$M=16.7$, 95\% CI $[16.6, 16.8]$}
\newcommand{\OcTime}[0]{$M=6.1$, 95\% CI $[6.0, 6.1]$}
\newcommand{\ThTime}[0]{$M=7.8$, 95\% CI $[7.7, 8.0]$}

%% AUTOMATICALLY GENERATED -- DO NOT EDIT!
\newcommand{\BqAccuracy}[0]{$M=95.3\%$, 95\%\ \mathrm{CI}\ $[95.1\%, 95.5\%]$}
\newcommand{\BqpAccuracy}[0]{$M=95.3\%$, 95\%\ \mathrm{CI}\ $[95.1\%, 95.5\%]$}
\newcommand{\ExpAccuracy}[0]{$M=88.2\%$, 95\%\ \mathrm{CI}\ $[87.9\%, 88.6\%]$}
\newcommand{\ExpaAccuracy}[0]{$M=86.6\%$, 95\%\ \mathrm{CI}\ $[86.1\%, 87.1\%]$}
\newcommand{\ExpbAccuracy}[0]{$M=89.9\%$, 95\%\ \mathrm{CI}\ $[89.4\%, 90.3\%]$}
\newcommand{\GsAccuracy}[0]{$M=100.0\%$, 95\%\ \mathrm{CI}\ $[99.7\%, 100.0\%]$}
\newcommand{\HcAccuracy}[0]{$M=59.7\%$, 95\%\ \mathrm{CI}\ $[59.2\%, 60.2\%]$}
\newcommand{\OcAccuracy}[0]{$M=100.0\%$, 95\%\ \mathrm{CI}\ $[100.0\%, 100.0\%]$}
\newcommand{\ThAccuracy}[0]{$M=100.0\%$, 95\%\ \mathrm{CI}\ $[100.0\%, 100.0\%]$}

%% AUTOMATICALLY GENERATED -- DO NOT EDIT!
\newcommand{\BqNumChance}[0]{12}
\newcommand{\BqpNumChance}[0]{14}
\newcommand{\ExpNumChance}[0]{62}
\newcommand{\HcNumChance}[0]{312}
\newcommand{\OcNumChance}[0]{zero}
\newcommand{\ThNumChance}[0]{zero}

%% AUTOMATICALLY GENERATED -- DO NOT EDIT!
\newcommand{\ExpTrialAccuracyCorr}[0]{$\rho=0.66$, 95\%\ \mathrm{CI}\ $[0.57, 0.73]$}
\newcommand{\ExpaTrialAccuracyCorr}[0]{$\rho=0.52$, 95\%\ \mathrm{CI}\ $[0.36, 0.66]$}
\newcommand{\ExpbTrialAccuracyCorr}[0]{$\rho=0.16$, 95\%\ \mathrm{CI}\ $[-0.02, 0.34]$}

%% AUTOMATICALLY GENERATED -- DO NOT EDIT!
\newcommand{\ExpTrialTimeCorr}[0]{$\rho=-0.76$, 95\%\ \mathrm{CI}\ $[-0.81, -0.69]$}
\newcommand{\ExpaTrialTimeCorr}[0]{$\rho=-0.54$, 95\%\ \mathrm{CI}\ $[-0.67, -0.39]$}
\newcommand{\ExpbTrialTimeCorr}[0]{$\rho=-0.46$, 95\%\ \mathrm{CI}\ $[-0.60, -0.30]$}

%% AUTOMATICALLY GENERATED -- DO NOT EDIT!
\newcommand{\BqTimeCorr}[0]{$r=0.28$, 95\% CI $[0.21, 0.36]$}
\newcommand{\BqpTimeCorr}[0]{$r=0.34$, 95\% CI $[0.26, 0.41]$}
\newcommand{\HcTimeCorr}[0]{$r=0.09$, 95\% CI $[-0.00, 0.17]$}
\newcommand{\OcTimeCorr}[0]{$r=0.57$, 95\% CI $[0.52, 0.61]$}
\newcommand{\ThTimeCorr}[0]{$r=0.43$, 95\% CI $[0.37, 0.49]$}

%% AUTOMATICALLY GENERATED -- DO NOT EDIT!
\newcommand{\BqAccuracyCorr}[0]{$r=0.23$, 95\% CI $[0.16, 0.30]$}
\newcommand{\BqpAccuracyCorr}[0]{$r=0.15$, 95\% CI $[0.08, 0.21]$}
\newcommand{\HcAccuracyCorr}[0]{$r=0.24$, 95\% CI $[0.17, 0.31]$}
\newcommand{\OcAccuracyCorr}[0]{$r=0.06$, 95\% CI $[-0.04, 0.16]$}
\newcommand{\ThAccuracyCorr}[0]{$r=0.06$, 95\% CI $[-0.04, 0.16]$}

%% AUTOMATICALLY GENERATED -- DO NOT EDIT!
\newcommand{\ExpAccuracyCorr}[0]{$r=0.57$, 95\% CI $[0.52, 0.61]$}
\newcommand{\ExpTimeCorr}[0]{$r=0.74$, 95\% CI $[0.66, 0.78]$}

%% AUTOMATICALLY GENERATED -- DO NOT EDIT!
\newcommand{\BqThetaTimeCorrFlipped}[0]{$\rho=0.63$, 95\%\ \mathrm{CI}\ $[0.55, 0.69]$}
\newcommand{\BqThetaTimeCorrSame}[0]{$\rho=0.63$, 95\%\ \mathrm{CI}\ $[0.55, 0.69]$}
\newcommand{\BqpThetaTimeCorrFlipped}[0]{$\rho=0.62$, 95\%\ \mathrm{CI}\ $[0.55, 0.69]$}
\newcommand{\BqpThetaTimeCorrSame}[0]{$\rho=0.61$, 95\%\ \mathrm{CI}\ $[0.53, 0.68]$}
\newcommand{\ExpThetaTimeCorrFlipped}[0]{$\rho=0.50$, 95\%\ \mathrm{CI}\ $[0.41, 0.58]$}
\newcommand{\ExpThetaTimeCorrSame}[0]{$\rho=0.68$, 95\%\ \mathrm{CI}\ $[0.61, 0.73]$}
\newcommand{\HcThetaTimeCorrFlipped}[0]{$\rho=0.18$, 95\%\ \mathrm{CI}\ $[0.03, 0.32]$}
\newcommand{\HcThetaTimeCorrSame}[0]{$\rho=0.17$, 95\%\ \mathrm{CI}\ $[0.03, 0.31]$}
\newcommand{\OcThetaTimeCorrFlipped}[0]{$\rho=0.99$, 95\%\ \mathrm{CI}\ $[0.99, 0.99]$}
\newcommand{\OcThetaTimeCorrSame}[0]{$\rho=0.99$, 95\%\ \mathrm{CI}\ $[0.99, 0.99]$}
\newcommand{\ThThetaTimeCorrFlipped}[0]{$\rho=0.56$, 95\%\ \mathrm{CI}\ $[0.47, 0.65]$}
\newcommand{\ThThetaTimeCorrSame}[0]{$\rho=0.58$, 95\%\ \mathrm{CI}\ $[0.49, 0.66]$}

%% AUTOMATICALLY GENERATED -- DO NOT EDIT!
\newcommand{\BqThetaAccuracyCorrFlipped}[0]{$\rho=-0.51$, 95\% CI $[-0.60, -0.41]$}
\newcommand{\BqThetaAccuracyCorrSame}[0]{$\rho=-0.49$, 95\% CI $[-0.58, -0.39]$}
\newcommand{\BqpThetaAccuracyCorrFlipped}[0]{$\rho=-0.56$, 95\% CI $[-0.64, -0.47]$}
\newcommand{\BqpThetaAccuracyCorrSame}[0]{$\rho=-0.51$, 95\% CI $[-0.60, -0.41]$}
\newcommand{\ExpThetaAccuracyCorrFlipped}[0]{$\rho=-0.36$, 95\% CI $[-0.46, -0.26]$}
\newcommand{\ExpThetaAccuracyCorrSame}[0]{$\rho=-0.77$, 95\% CI $[-0.81, -0.72]$}
\newcommand{\HcThetaAccuracyCorrFlipped}[0]{$\rho=-0.49$, 95\% CI $[-0.59, -0.38]$}
\newcommand{\HcThetaAccuracyCorrSame}[0]{$\rho=-0.50$, 95\% CI $[-0.60, -0.39]$}
\newcommand{\OcThetaAccuracyCorrFlipped}[0]{$\rho=0.00$, 95\% CI $[-0.17, 0.17]$}
\newcommand{\OcThetaAccuracyCorrSame}[0]{$\rho=0.00$, 95\% CI $[-0.17, 0.17]$}
\newcommand{\ThThetaAccuracyCorrFlipped}[0]{$\rho=0.00$, 95\% CI $[-0.17, 0.17]$}
\newcommand{\ThThetaAccuracyCorrSame}[0]{$\rho=0.00$, 95\% CI $[-0.17, 0.17]$}



\title{What to simulate? Inferring the right direction for mental
  rotation}
 
\author{{\large \bf Jessica B. Hamrick (jhamrick@berkeley.edu)} \\
  {\large \bf Thomas L. Griffiths (tom\_griffiths@berkeley.edu)} \\
  Department of Psychology, University of California, Berkeley, CA
  94720 USA}

\begin{document}

\maketitle


\begin{abstract}
  When people use mental imagery, how do they decide \textit{which}
  images to generate? To answer this question, we explored how mental
  simulation should be used in the classic psychological task of
  determining if two images depict the same object in different
  orientations \cite{Shepard1971}. Through a rational analysis of
  mental rotation, we formalized four models and compared them to
  human performance. We found that three models based on previous
  hypotheses in the literature were unable to account for several
  aspects of human behavior. The fourth, which is a model based on the
  idea of \textit{active sampling} \cite<e.g.,>{Gureckis:2012gu},
  provides plausible explanations where the other models do not. Based
  on these results, we suggest that the question of ``what to
  simulate?'' is more difficult than has previously been assumed, and
  that an active learning approach holds promise for uncovering the
  answer.

  \textbf{Keywords:} mental rotation, computational modeling
\end{abstract}

\section{Introduction}

One of the most astonishing cognitive feats is our ability to
envision, manipulate, and plan with objects---all without actually
perceiving them. This \textit{mental simulation} has been widely
studied, including an intense debate about the underlying
representation of mental images \cite<e.g.,>{Kosslyn:2009tj,
  Pylyshyn:2002vk}. But this debate hasn't addressed one of the most
fundamental questions about mental simulation: how people decide
\textit{what} to simulate.

Mental rotation provides a simple example of the decision problem
posed by simulation.  In the classic experiment by
\citeA{Shepard1971}, participants viewed images of three-dimensional
objects and had to determine whether the images depicted the same
object (which differed by a rotation) or two separate objects (which
differed by a reflection and a rotation). They found that people's
response times (RTs) had a strong linear correlation with the minimum
angle of rotation, a result which led to the conclusion that people
solve this task by ``mentally rotating'' the objects until they are
congruent.  However, this explanation leaves several questions
unanswered. How do people know the axis around which to rotate the
objects? If the axis is known, how do people know which direction to
rotate the objects?  And finally, how do people know how long to
rotate?

In this paper, we explore these questions through rational analysis
\cite{Marr:1983to,anderson90,Shepard:1987tt} and compare four models
of mental rotation. We begin the paper by discussing the previous
literature on mental imagery. Next, we outline computational- and
algorithmic-level analyses of the problem of mental rotation.  We then
describe a behavioral experiment based on the classic mental rotation
studies \cite<e.g.,>{Cooper:1975wp}, and compare the results of our
experiment with each of the models. We conclude with a discussion of
the strengths and weaknesses of each model, and lay out directions for
future work.

\section{Modeling mental rotation}

\begin{figure*}[t]
  \begin{center}
    \includegraphics[width=\textwidth]{../../figures/D/model-traces.pdf}
    \caption{\textbf{Example stimuli and model behavior.} On the left
      is a ``flipped'' stimulus pair with a rotation of
      $120^\circ$. The grey plots show the actions that each model
      took for this stimulus. The color of each point (action)
      indicates whether the model's current mental image was generated
      under the hypothesis that the objects are the same (red) or
      flipped (blue). The similarity functions for each of these
      hypotheses are shown on the right.}
    \label{fig:model-traces}
  \end{center}
\end{figure*}

Previous models of mental rotation have largely focused on the
representation of mental images, rather than how people decide
\textit{which} mental images to generate. \citeA{Kosslyn:1977tv}
proposed a model of the mental imagery buffer, but did not say
\textit{how} it should be used. Similarly, \citeA{Julstrom:1985va} and
\citeA{Glasgow:1992tj} were mostly concerned with modeling the
representational format underlying imagery. Although
\citeA{Anderson1978} emphasized the importance of considering both
representation and process, he dismissed the problem of determining
the direction of rotation as a ``technical difficulty''.

The only models (of which the authors are aware) that seriously
attempted to address the decision of \textit{what} to simulate are
those by \citeA{Funt:1983wn} and \citeA{Just:1985uu}. In both of these
models, the axis and direction of rotation are computed prior to
performing the rotation. One object is then rotated through the target
rotation, and is checked against the other object for
congruency. However, this approach assumes that the corresponding
points on the two objects can be easily identified, which is not
necessarily the case.  Indeed, the state-of-the-art in computer vision
suggests that there is more to this problem than simply checking for
congruency, particularly when shapes are complex or when the shapes
might not be exactly the same
\cite<e.g.,>{Belongie:2002tj,Sebastian:2003vm}. Additionally, recent
research shows that when performing \textit{physical} rotations,
people do not fully rotate until congruency is reached; they may even
rotate \textit{away} from initially near perfect matches
\cite{Gardony:2013gn}.

If people are not computing the rotation beforehand, what might they
be doing? To answer this question, we perform a rational analysis of
the problem of mental rotation
\cite{Marr:1983to,anderson90,Shepard:1987tt}. At the computational
level, we can say that the \textit{problem} is to determine which
spatial transformations an object has undergone based on two images of
that object (which do not include information about point
correspondences). At the algorithmic level, we are constrained by the
notion that mental images must be transformed in an analog manner (or
in a way that is at least approximately analog), and that mental
images are time-consuming and effortful to generate. Thus, the
\textit{goal} is to make this determination while performing a minimum
amount of computation (i.e., as few rotations as possible).

The original ``congruency'' hypothesis \cite{Shepard1971} actually is
a rational solution to this problem, in the sense that the smallest
amount of computation coincides with rotating through the minimum
angle. However, it violates the constraint that we do not know the
points of correspondence between the images, which is exactly what
necessitates the use of imagery. Noting that a rational solution need
not necessarily maintain a single trajectory of rotation, we explore
an alternative model, which---rather than explicitly computing the
angle of rotation---engages in an \textit{active sampling} strategy.

Active sampling is the idea that when initial information is
incomplete, people will gather new information in a manner that
increases certainty about the problem space. This notion has recently
gained support in several areas of cognitive science
\cite<e.g.,>{Gureckis:2012gu}, including other spatial domains
\cite{Juni:2011vo}. In the case of mental rotation, if people do not
know the minimum angle beforehand, then actively performing rotations
may be the best way to gather evidence about the similarity between
the observed shapes.

\section{How should we rotate?}

In this section, we formalize our rational analysis and propose four
models of mental rotation: one based on existing models; two which are
extensions of the first but with relaxed assumptions; and one based on
the active sampling approach.

\subsection{Computational-level analysis}

Formally, we denote the shapes as $X_a$ and $X_b$ and assume $X_b$ is
generated by a transformation of $X_a$, i.e. $X_b=f(X_a, \theta, h)$,
where $\theta$ is a rotation, $\hi$ is the hypothesis that the images
depict the same object, and $\hf$ is the hypothesis that the images
depict mirror-image objects. The posterior probability of each
hypothesis given the observed shapes is then: $p(h\ \vert\ X_a, X_b)
\propto \int p(X_b\ \vert\ X_a, \theta, h)p(h)p(\theta)\dif\theta$,
where $p(X_b\ \vert\ X_a, \theta, h)$ is the probability that $X_b$
was generated from $X_a$. Because we ultimately want to determine
which hypothesis is more likely, the quantity of interest is a ratio
$\ell:=p(\hi\ \vert\ X_a, X_b) / p(\hf\ \vert\ X_a, X_b)$ which
(assuming that all rotations are equally likely) is equivalent to:
\begin{equation}
  \ell = \frac{\left(\int p(X_b\ \vert\ X_a, \theta, \hi)\dif\theta\right)p(\hi)}{\left(\int p(X_b\ \vert\ X_a, \theta, \hf)\dif\theta\right)p(\hf)}.
  \label{eq:lh-ratio}
\end{equation}
If $\ell > 1$, then we accept the hypothesis that the images depict
the same object ($\hi$); if $\ell < 1$, then we accept the hypothesis
that the images depict flipped objects ($\hf$).

\subsection{Algorithmic constraints}

\begin{figure*}[t]
  \begin{center}
    \includegraphics[width=\textwidth]{../../figures/D/response_time_accuracy.pdf}
    \caption{\textbf{Response time and accuracy comparison.} Top: RT
      of correct responses as a function of the minimum angle of
      rotation. Bottom: accuracy as a function of the minimum angle of
      rotation. All error bars are 95\% confidence intervals.}
    \label{fig:response-time-accuracy}
  \end{center}
\end{figure*}

We represent a shape of $N$ vertices with a $N\times 2$ coordinate
matrix $\mathbf{X}=[\mathbf{x}_1, \ldots{}, \mathbf{x}_N]$, and denote
the rotation and/or reflection transformation as $f(\mathbf{X}, h,
\theta):=\mathbf{X}\F_h^T\R^T$, where $\R$ is a rotation matrix, and
$\F_h$ is either the identity matrix $\I$ (when $\hi$) or a reflection
matrix across the $y$-axis (when $\hf$).

We define $p(\Xb\ \vert\ \Xa, \theta, h)$ to be the similarity between
$\Xb$ and a transformation of $\Xa$: $p(\Xb\ \vert\ \Xa, \theta, h):=
S(\Xb, f(\Xa, h, \theta))$.  We do not know which vertices of $\Xb$
correspond to which vertices of $\Xa$, so the similarity $S$ must
marginalize over the set of possible mappings. For brevity, let
$\mathbf{X}_m=\M\cdot{}f(\Xa, h, \theta)$ where $\M$ is a permutation
matrix. Then:
\begin{equation}
  S(\Xb, f(\Xa, h, \theta)):=\frac{1}{2N} \sum_{\M} \prod_{n=1}^N \mathcal{N}(\mathbf{x}_{bn}\ \vert \ \mathbf{x}_{mn}, \I\sigma_S^2),
  \label{eq:similarity}
\end{equation}
where $2N$ is the total number of possible mappings,\footnote{It is
  $2N$ and not $N^2$ because, in polar coordinates, vertices are
  always connected to their two nearest neighbors in the $\theta$
  dimension.}and $\sigma_S^2=0.15$ is the variance of the
similarity. Example similarity curves are shown in
Figure~\ref{fig:model-traces}.

Additionally, we assume that the observed shapes can only be
transformed by a small amount at a time, and each transformation takes
a non-negligible amount of time. Specifically, if the current mental
image is $\Xt$, then:
\begin{equation}
  \mathbf{X}_{t+1} = \left\{ \begin{array}{ll}
      f(\Xt, 0, \epsilon) &\mbox{ rotate by $\epsilon$ radians,} \\
      f(\Xt, 1, 0) &\mbox{ flip,} \\
      f(\Xa, 0, 0) &\mbox{ reset, or} \\
      f(\Xa, 1, 0) &\mbox{ reset and flip,} \\
    \end{array} \right.
  \label{eq:actions}
\end{equation}
where $\epsilon\sim \left|\mathcal{N}(0, \sigma_\epsilon^2)\right|$
and $\sigma_\epsilon^2$ is the variance of the step size.

To summarize, we approximate the likelihood term of
Equation~\ref{eq:lh-ratio} using the similarity function defined in
Equation~\ref{eq:similarity}. Because we assume mental rotations are
performed sequentially, this similarity can only be computed for the
actions listed in Equation~\ref{eq:actions}.

\subsection{Specific models of mental rotation}

In order to approximate Equation~\ref{eq:lh-ratio} using samples of
the similarity function, we must decide \textit{which} places to
sample and \textit{when} stop sampling. The models below differ in how
they make these decisions; see Figure~\ref{fig:model-traces} for an
example.

\subsubsection{Oracle model}

One hypothesis is that people compute the direction and extent of
rotation beforehand using \textit{a priori} knowledge of the
correspondence between points in the images
\cite{Funt:1983wn,Just:1985uu}.  To reflect this hypothesis, we
created an ``oracle'' model which is told which points on each shape
correspond. From that correspondence, it computes the correct rotation
and rotates through it.

To determine the correct rotation, we solve for the rotation matrix by
computing $(\Xa \F_h^T)_\mathrm{left}^{-1}\cdot{}\Xb$, where
$(\Xa\F_h^T)_\mathrm{left}^{-1}$ is the left inverse of
$\Xa\F_h^T$. We then check each $h$ to see if the computation produces
a valid rotation matrix; the $h$ that does is the correct
hypothesis. This gives us the true value of $\theta$, so
Equation~\ref{eq:lh-ratio} becomes a generalized likelihood ratio
test, where $\theta$ is set to the MLE value, rather than being
marginalized:
\begin{equation}
  \ell = \frac{\max_\theta p(\Xb\ \vert\ \Xa, \theta, \hi)p(\hi)}{\max_\theta p(\Xb\ \vert\ \Xa, \theta, \hf)p(\hf)}.
  \label{eq:mle-lh-ratio}
\end{equation}

\subsubsection{Threshold model}

What if the point correspondences are unknown? A model which obeys
this constraint could first rotate in the direction that increases
local similarity, and then stop rotating once a ``match'' is found, as
determined by a threshold on the similarity $S$. If all rotations have
been exhausted, then flip, and try rotating again. We assume that the
locations where $S$ is greater than the threshold correspond to the
true $\theta$ (or points near the true $\theta$). So, as with the
\Oc{} model, we use Equation~\ref{eq:mle-lh-ratio}.

\subsubsection{Hill Climbing model}

\begin{figure*}[t]
  \begin{center}
    \includegraphics[width=\textwidth]{../../figures/D/response_time_scatters.pdf}
    \caption{\textbf{Model vs. human RTs.} Each subplot shows the
      z-scored model RTs ($x$-axis) vs. the z-scored human RTs
      ($y$-axis). Pearson correlations are shown beneath each
      subplot. The dotted lines are $x=y$.}
    \label{fig:human-model-scatters}
  \end{center}
\end{figure*}

In the current formulation of the problem, choosing the threshold is
straightforward because we know both the exact geometry of the shapes
and that a linear transformation exists which will perfectly align
them. However, this choice is not always clear \textit{a priori}, as
the global optimum depends on many factors (e.g., shape complexity,
dimensionality, perceptual uncertainty, and whether the shapes are
identical).  One way to deal with the problem of choosing a threshold
would simply be to perform an exhaustive search or use a global
optimization strategy. However, these strategies would not result in
the linear RT found by \citeA{Shepard1971}. A second alternative might
instead just perform a Hill Climbing (\Hc{}) search for a
\textit{local} maximum. Thus, as with the \Oc{} and \Th{} models, we
use Equation~\ref{eq:mle-lh-ratio}.

\subsubsection{Bayesian Quadrature model}

While the previous few models all focused on \textit{searching} for
the global maximum, we need only approximate
Equation~\ref{eq:lh-ratio} with a set of sequential rotations. Based
on evidence that people use \textit{active sampling}
\cite<e.g.,>{Gureckis:2012gu}, we hypothesize a model that actively
monitors the usefulness of the rotations it is performing. Instead of
searching for a maximum, we maintain a probability distribution over
our \textit{estimate} of Equation~\ref{eq:lh-ratio}, and then sample
actions which are expected to improve that estimate.  This strategy
has several benefits. First, it does not make assumptions about the
scale of the similarity function. Second, by choosing to sample places
which are informative, this method implicitly minimizes the amount of
rotation.

Formally, we denote $Z_h$ as our estimate of the likelihood for
hypothesis $h$, and write its distribution as: $p(Z_h) = \int
\left[\int S(\Xb, f(\Xa, \theta, h))p(\theta)\dif\theta\right]
p(S)\dif S$, where $S$ is the similarity function, and $p(S)$ is a
prior over similarity functions.  This method of estimating an
integral is known in the machine-learning literature as
\textit{Bayesian Quadrature} \cite{Diaconis:1988uo,Osborne:2012tm}, or
BQ.

Denoting $S_h=S(\Xb, f(\Xa, \theta, h))$, we first place a
\textit{Gaussian Process} \cite{Rasmussen:2006vz}, or GP, prior on the
log of $S_h$ in order to enforce positivity after it is exponentiated,
i.e. $\mathbb{E}[Z_h] \approx \int
\exp(\mu_h(\theta))p(\theta)\dif\theta$, where $\mu_h:=\mu(\log S_h)$
is the mean of the log-GP \cite{Osborne:2012tm}.  However, this
integral is still intractable. To approximate it, we fit a second GP
over points sampled from the log-GP, which we denote as
$\bar{S}_h:=\exp(\mu_h)$. Then, from \citeA{Duvenaud:2013td}, we have
$\mathbb{E}[Z_h] \approx \int \bar{\mu}_h(\theta)p(\theta)\dif\theta$
and $\mathbb{V}(Z_h) \approx \iint \mathrm{Cov}_h(\theta,
\theta^\prime)\bar{\mu}_h(\theta)\bar{\mu}_h(\theta^\prime)p(\theta)p(\theta^\prime)\dif\theta\dif\theta^\prime$,
where $\bar{\mu}_h:=\mu(\bar{S}_h)$ is the mean of the second GP, and
$\mathrm{Cov}_h:=\mathrm{Cov}(\log S_h)$ is the covariance of the
log-GP.

Assuming independence, we can now write $p(Z_h)\approx\mathcal{N}(Z_h\
\vert\ \mathbb{E}[Z_h], \mathbb{V}(Z_h))$, which gives us a
distribution over the likelihood ratio in Equation~\ref{eq:lh-ratio}:
\begin{equation}
p(\ell)\approx\frac{\mathcal{N}(Z_0\ \vert\ \mathbb{E}[Z_0], \mathbb{V}(Z_0))\ p(\hi)}{\mathcal{N}(Z_1\ \vert\ \mathbb{E}[Z_1], \mathbb{V}(Z_1))\ p(\hf)}.
\end{equation}
This distribution cannot easily be calculated; however, we really are
just interested in whether $Z_0>Z_1$ or $Z_1>Z_0$. Thus, rather than
computing $p(\ell)$, we use $Z_D=Z_0-Z_1$ and compute:
$p(Z_D)\propto\mathcal{N}(\mathbb{E}[Z_0] - \mathbb{E}[Z_1],
\mathbb{V}(Z_0) + \mathbb{V}(Z_1))$.  We then sample new observations
until we are at least 95\% confident that $Z_D\neq 0$. In other words,
when $p(Z_D<0)<0.025$, we accept $\hi$, and when $p(Z_D<0)>0.975$, we
accept $\hf$. To determine which points to sample, we can compute the
expected variance of $Z_h$ if we sampled a new observation at
$\theta_a$. From \citeA{Osborne:2012tm}, we compute the expected
variance as $\mathbb{E}[\mathbb{V}(Z_h|\theta_a)]=\mathbb{V}(Z_h) +
\mathbb{E}[Z_h] - \int \mathbb{E}[Z_h|\theta_{a}]^2
\mathcal{N}(\mu_h(\theta_a), \mathrm{Cov}_h(\theta_a,
qc\theta_a))\dif\log S_h(\theta_a)$ for each of the actions in
Equation~\ref{eq:actions}, and then choose the action with the lowest
value.

\section{Methods}

To evaluate the models described previously, we ran a behavioral
experiment based on classic mental rotation studies
\cite<e.g.>{Shepard1971, Cooper:1975wp}.

\subsubsection{Stimuli}

We randomly generated 20 shapes of five or six vertices (e.g.,
Figure~\ref{fig:model-traces}). For each shape, we computed 20
``same'' and 20 ``flipped'' stimuli pairs, with 18 rotations
($\theta$) spaced at $20^\circ$ increments between $0^\circ$ and
$360^\circ$ (with $0^\circ$ and $180^\circ$ repeated twice, in order
to gather an equal number of responses for each angle between
$0^\circ$ and $180^\circ$). ``Same'' pairs were created by rotating
$\Xa$ by $\theta$; ``flipped'' pairs were first reflected $\Xa$ across
the $y$-axis, then rotated by $\theta$.

We generated five additional shapes to be used in a practice block of
10 trials. Across these trials, there was one ``flipped'' and one
``same'' repetition of each shape and each angle ($60^\circ$,
$120^\circ$, $180^\circ$, $240^\circ$, or $300^\circ$) such that no
shape was presented at the same angle twice. We also generated a sixth
shape to include with the instructions.  This shape had both a
``flipped'' and ``same'' version, each rotated to $320^\circ$.


\subsubsection{Participants and Design}

We recruited 247 participants on Amazon's Mechanical Turk using the
psiTurk experiment framework \cite{Mcdonnell12}. Each participant was
paid \$1.00 for approximately 15 minutes of work, consisting of one
block of 10 practice trials followed by two blocks of 100 experiment
trials. Within a block, trials were presented in a random order.

All participants saw the same 10 practice trials as described
above. There were 720 unique experimental stimuli (20 shapes $\times$
18 angles $\times$ 2 reflections), though because stimuli with
rotations of $0^\circ$ or $180^\circ$ were repeated twice, there were
800 total experimental stimuli. These stimuli were split across eight
conditions in the following manner: first, stimuli were split into
four blocks of 200 trials. Within each block, each shape was repeated
ten times and each rotation was repeated ten times (five ``same'',
five ``flipped''), such that across all blocks, each stimulus appeared
exactly once. Each block was then split in half. Participants
completed two half-blocks (not necessarily both from the original full
block).

\subsubsection{Procedure}

\begin{figure*}[t]
  \begin{center}
    \includegraphics[width=\textwidth]{../../figures/D/response_time_histograms.pdf}
    \caption{\textbf{Response time histograms.} Each subplot shows the
      distribution of RTs on correct trials for people and the
      models.}
    \label{fig:histograms}
  \end{center}
\end{figure*}

Participants were given the following instructions while being shown
an example ``same'' pair and an example ``flipped'' pair: \textit{``On
  each trial, you will see two images. Sometimes, they show the
  \textbf{same} object. Other times, the images show \textbf{flipped}
  objects. The task is to determine whether the two images show the
  \textbf{same} object or \textbf{flipped} objects.''}

On each trial, participants were instructed to press the `b' key to
begin and to focus on the fixation cross that appeared for 750ms
afterwards. The two images were then presented side-by-side, each at
300px $\times$ 300px, and participants could press `s' to indicate
they thought the images depicted the ``same'' object, or `d' to
indicate they thought the images depicted ``flipped'' objects.

While there was no limit on RT, participants were urged to respond as
quickly as possible while still being accurate. Specifically, we asked
them to aim for at least 85\% accuracy in the experimental blocks, and
provided a counter to keep them informed of their score.

\section{Results}

Out of the 247 participants, 200 (81\%) were included in our
analyses. Of the other 47, we excluded 10 (4\%) because of an
experimental error, 6 (2.4\%) because they had already completed a
related experiment, and 31 (12.6\%) because they failed a
comprehension check. Comprehension was defined as correctly answering
at least 85\% of stimuli with a minimum rotation of $0^\circ$ or
$20^\circ$.  Additionally, we excluded 82 trials for which the RT was
either less than 100ms or greater than 20s.

For each model, we ran 50 samples for each of the 800 experimental
stimuli. The step size parameter ($\sigma_\epsilon$) was fit to human
RTs for each of the models, resulting in $\sigma_\epsilon=0.6$ for the
\Th{} and \Bq{} models and $\sigma_\epsilon=0.1$ for the \Oc{} and
\Hc{} models. We also ran the models under two different priors,
$p(h=0)=0.5$ (the ``equal'' prior) and $p(h=0)=0.55$ (the ``unequal''
prior). This only had a major effect on the stopping criteria for the
\Bq{} model.

\subsubsection{General analysis}

For analyses of RT, confidence intervals around harmonic means were
computed using a bootstrap analysis of 10000 bootstrap samples
(sampled with replacement). RTs were only considered for correct
responses.  We also used a bootstrap analysis of 10000 bootstrap
samples to compute the confidence intervals around both Spearman
($\rho$) and Pearson ($r$) correlations.  Unless otherwise specified,
all correlations were computed over 720 datapoints (one for each
stimulus). For analyses of accuracy, confidence intervals were
computed from a binomial proportion with a Jeffrey's beta prior.  To
test if participants or a model was above chance on a particular
stimulus, we used the same binomial proportion and tested whether
$p\left(p(\textrm{correct})\leq 0.5\right)\leq \frac{0.05}{720}$,
where $\frac{1}{720}$ is a Bonferroni correction for multiple
comparisons.

\subsubsection{Human}

The average RT across all correctly-judged stimuli was \ExpTime{} ms;
the full histogram of RTs can be seen in
Figure~\ref{fig:histograms}. The minimum angle of rotation was
significantly rank-order (Spearman) correlated with average
per-stimulus RTs, both for ``flipped'' (\ExpThetaTimeCorrFlipped{})
and ``same'' pairs (\ExpThetaTimeCorrSame{}). While this replicates
the general result of previous experiments
\cite<e.g.,>{Shepard1971,Cooper:1975wp}, our results are not as linear
(Figure~\ref{fig:response-time-accuracy}).

The average accuracy across all stimuli was \ExpAccuracy{}, though
there were \ExpNumChance{} stimuli (out of 720) for which people were
not above chance.  The minimum angle was also correlated with
participants' average per-stimulus accuracy, though significantly more
so for ``same'' pairs (\ExpThetaAccuracyCorrSame{}) than ``flipped''
pairs (\ExpThetaAccuracyCorrFlipped{}). This is the same result found
both by \citeA{Cooper:1975wp} and \citeA{Gardony:2013gn}.

There was a significant effect of trial number both on RT
(\ExpTrialTimeCorr{}) and on accuracy (\ExpTrialAccuracyCorr{}),
though the effect on accuracy was not significant during the second
half of the experiment (\ExpaTrialAccuracyCorr{} for the first half
vs. \ExpbTrialAccuracyCorr{} for the second half). These practice
effects may have contributed to the not-quite-linearity of the human
RTs; future work will need to collect more data per participant to
mitigate this effect.

\subsubsection{\Oc{} model}

The number of actions taken by the \Oc{} model was perfectly
correlated with the minimum angle of rotation
(Figure~\ref{fig:response-time-accuracy}). The \Oc{} model was the
best fit to human RTs, with a correlation of \OcTimeCorr{}
(Figure~\ref{fig:human-model-scatters}).  However, the distribution of
response times did not match that of people
(Figure~\ref{fig:histograms}). Moreover, the \Oc{} model was 100\%
accurate, and therefore could not explain the effect of rotation on
people's accuracy.

\subsubsection{\Th{} model}

There was an overall monotonic relationship between the minimum angle
of rotation and the number of actions taken by the \Th{} model
(Figure~\ref{fig:response-time-accuracy}). However, the relationship
between angle and RT for individual shapes was usually not monotonic
(e.g., Figure~\ref{fig:response-time-stimulus}).  The \Th{} model was
able to explain a moderate amount of the variance in human RTs, with a
correlation of \ThTimeCorr{}
(Figure~\ref{fig:human-model-scatters}). Like the \Oc{} model,
however, the overall distribution of its RTs did not match that of
people (Figure~\ref{fig:histograms}). Additionally, the \Th{} model
had 100\% accuracy, and thus did not exhibit a relationship between
minimum angle and accuracy.

As noted, we fit $\sigma_\epsilon=0.6$ for the \Th{} model. This has
the interesting effect of actually causing the \Th{} model to
\textit{over}rotate (see Figure~\ref{fig:model-traces}). This is
because the step size is large enough that it sometimes misses the
global maximum, and must do another full rotation to find it. However,
when the step size is smaller, the fit to human data is slightly
worse.

\subsubsection{\Hc{} model}

The \Hc{} was the only model for which there was no monotonic
relationship between rotation and RT
(Figure~\ref{fig:response-time-accuracy}). Moreover, the \Hc{} model
was barely above chance (\HcAccuracy{}) and there were \HcNumChance{}
individual stimuli for which it was not above chance. The \Hc{} model
was not a particularly good predictor of human RTs (\HcTimeCorr{}), as
shown in Figure~\ref{fig:human-model-scatters}. It was a moderately
good predictor of human accuracy (\HcAccuracyCorr{}).

\subsubsection{\Bq{} model}

Like the \Oc{} and \Th{} models, there was an overall monotonic
relationship between rotation and the number of steps taken by the
\Bq{} model (Figure~\ref{fig:response-time-accuracy}). Unlike the
\Th{} model, this relationship existed for individual shapes as well
(e.g., Figure~\ref{fig:response-time-stimulus}).  The \Bq{} model
explained variance in human RTs about as well as the \Th{} model
(Figure~\ref{fig:human-model-scatters}), with a correlation of
\BqTimeCorr{} for the equal prior and \BqpTimeCorr{} for the unequal
prior, and the RT distribution from the \Bq{} model had the same
overall shape as that of people (Figure~\ref{fig:histograms}).

The \Bq{} model was quite accurate overall (equal prior:
(\BqAccuracy{}; unequal prior: \BqpAccuracy{}). With the equal prior,
there were \BqNumChance{} stimuli for which it was not above chance;
with the unequal prior, there were \BqpNumChance{}. The \Bq{} model
had about the same correlation with people's accuracy as the \Hc{}
model (equal prior: \BqAccuracyCorr{}; unequal prior:
\BqpAccuracyCorr{}).

Because the \Bq{} model relies on Equation~\ref{eq:lh-ratio} for its
stopping criteria (as opposed to just finding a maximum), the prior
$p(h)$ had a fairly significant effect
(Figure~\ref{fig:response-time-accuracy}). With just a small bias of
$p(h=0)=0.55$, there was a clear separation in RTs for ``same'' versus
``flipped'' stimuli; this separation is similar to the trend also
observed in human RTs. While the prior also had an effect on accuracy,
this effect did not reflect the trends in human behavior.

\section{Discussion}

\begin{figure}[t]
  \begin{center}
    \includegraphics[width=0.48\textwidth]{../../figures/D/response_time_stimulus.pdf}
    \caption{\textbf{Typical RT curves for a single shape.}  These
      plots correspond to the shape shown in
      Figure~\ref{fig:model-traces}. Left: human curves are either
      linear (as with the ``same'' pairs), or linear and then flat (as
      with the ``flipped'' pairs). Middle: the \Th{} model does not
      have a reliably monotonic relationship with rotation. Right: the
      \Bq{} model is roughly linear.}
    \label{fig:response-time-stimulus}
  \end{center}
\end{figure}

We set out to answer the question of how people decide \textit{what}
to simulate when using mental imagery. Focusing on the specific case
of determining the direction and extent of mental rotation, we
formalized four models and compared their performance with the results
of a behavioral experiment.

The \Oc{} and \Th{} models were---in terms of correlations---the best
predictors of human RTs. However, both are somewhat unsatisfying
explanations because they rely on \textit{a priori} knowledge that
people are unlikely to have. Moreover, they offer no explanation of
several aspects of human behavior. First, their overall RT
distributions look nothing like people's
(Figure~\ref{fig:histograms}).  Second, they both are 100\% accurate,
and so cannot explain the systematic relationship between rotation and
human accuracy (Figure~\ref{fig:response-time-accuracy}). Third,
neither model can explain the difference in people's behavior on
``same'' and ``flipped'' stimuli.

In contrast, the \Bq{} model was nearly as good as the \Th{} model,
yet it makes no assumptions about people's \textit{a priori}
knowledge. Furthermore, the \Bq{} model matches people's behavior
better than the \Oc{} or \Th{} models in several ways. Its overall RT
histogram has the same general shape as people's
(Figure~\ref{fig:histograms}).  Moreover, a closer look shows that the
\Bq{} model maintains the monotonic relationship between angle and RT
even on individual stimuli, while the \Th{} model does not
(Figure~\ref{fig:response-time-stimulus}). Finally, the \Bq{} model's
adaptive stopping rule is sensitive to the prior, and thus provides a
possible explanation for why people are slower to respond on
``flipped'' stimulus pairs.

Thus, we suggest that the \Bq{} model offers the most promising
explanation of people's behavior on the mental rotation task to
date. While it is not a perfect account, there are several ways in
which it could be improved. For example, while we used holistic
rotations in this paper, there is evidence that people compare
individual features of shapes
\cite{Just1976,Yuille:1982tx}. Additionally, a different active
sampling approach could maintain a distribution over the location and
value of the global maximum, rather than over the integral. We intend
to explore these possibilities in future work, building upon the
foundation established in this paper and working towards a better
understanding of \textit{what} people choose to simulate.


%% TODO: Uncomment this for final version:

% \section{Acknowledgments}
% {\small This research was supported by ONR MURI grant number
% N00014-13-1-0341, and a Berkeley Fellowship awarded to JBH. }

\bibliographystyle{apacite}
\renewcommand{\bibliographytypesize}{\small}
\setlength{\bibleftmargin}{.125in}
\setlength{\bibindent}{-\bibleftmargin}
\bibliography{references}

\end{document}
