\documentclass{article} % For LaTeX2e
\usepackage{nips12submit_e,times}
%\documentstyle[nips12submit_09,times,art10]{article} % For LaTeX 2.09

\usepackage{amsmath, amsthm, amssymb}
% \usepackage{natbib}
\usepackage{graphicx}
\usepackage{wrapfig}

\pagestyle{empty}

% \usepackage{etoolbox}
% \makeatletter
% \patchcmd{\thebibliography}{%
%   \chapter*{\bibname}\@mkboth{\MakeUppercase\bibname}{\MakeUppercase\bibname}}{%
%   \subsubsection*{References}}{}{}
% \makeatother

\title{Optimal strategies in mental rotation tasks}
\author{Jessica B.~Hamrick\\
  Department of Psychology\\
  University of California, Berkeley\\
  Berkeley, CA 94720\\
  \texttt{jhamrick@berkeley.edu}}

% The \author macro works with any number of authors. There are two commands
% used to separate the names and addresses of multiple authors: \And and \AND.
%
% Using \And between authors leaves it to \LaTeX{} to determine where to break
% the lines. Using \AND forces a linebreak at that point. So, if \LaTeX{}
% puts 3 of 4 authors names on the first line, and the last on the second
% line, try using \AND instead of \And before the third author name.

\newcommand{\fix}{\marginpar{FIX}}
\newcommand{\new}{\marginpar{NEW}}

\nipsfinalcopy % Uncomment for camera-ready version

\begin{document}

\maketitle

% Your proposal should be one page long (single-spaced), and should have
% four sections:

% \begin{wrapfigure}{r}{0.4\textwidth}
\begin{figure}[h]
  \begin{center}
    \includegraphics[width=0.95\textwidth]{../figures/shepard-rotation.png}
  \end{center}
  \caption{Classic mental rotation task from \cite{Shepard1971}.}
  \label{fig:mental-rotation}
% \end{wrapfigure}
\end{figure}

\section{Background}
% - Background. Identify the topic your project will explore, and
% briefly provide some of the context for your project, describing
% what previous research has found in this area.

% 1. The big picture
% 2. Our contribution
% 3. How we do it
% 4. What's in the paper


Consider the objects in Figure \ref{fig:mental-rotation}. In each
panel, are the two depicted objects identical (except for a rotation),
or distinct? When presented with this mental rotation task, people
default to a strategy in which they visualize one object rotating until
it is congruent with the other \cite{Shepard1971}. There is strong
evidence for such \textit{mental imagery} or \textit{mental
  simulation}: we can imagine three-dimensional objects in our minds
and manipulate them, to a certain extent, as if they were real
\cite{Kosslyn:2009tj}.

However, the use of mental simulation is predicated on determining
appropriate parameters to give the simulation, and people's cognitive
constraints may furthermore place limitations on the duration or
precision of simulation. One hypothesis for how these issues are
handled argues that people use a ``rational'' solution, meaning that
it is optimal under the given constraints
\cite{Lieder:2012wg,Vul:2009wy,Griffiths2012a}.

\section{Question}
% - Question. State the specific question you are going to examine in
% your final project.

In the case of the classic mental rotation task, we might ask: in what
direction should the object be rotated?  What are the requirements for
``congruency''? When should one stop rotating and accept the
hypothesis that the objects are different? In this project, I will
investigate the optimal computational solution to this task. Thus, the
specific question this project aims to answer is: what is the rational
computational solution to the mental rotation task, and does it
predict the people's behavior on the task?

\section{Method}
% - Method. Briefly describe the method you are going to use to try to
% answer this question. Give some of the details behind your
% experimental procedure, your approach to modeling or analyzing the
% data, your plans for analyzing the model, or the position you will
% take in your review.

We can formalize the mental rotation task as an instance of function
learning. Given two images $I_a$ and $I_b$, we wish to determine the
probability of seeing both images given the hypothesis ($h_1$) that
one is merely a rotation of the other: $\Pr(I_b, I_a|h_1)$. We can
generate some mental image $I_M$ using mental rotation, but this is a
costly operation and each $I_M$ must be computed sequentially (that
is, each $I_t$ must be generated from $I_{t-\delta}$, where $\delta$
is a small angle). Let us define the pixel-space similarity between
$I_b$ and $I_M$ as $S(I_b| I_M)$. Then:

\begin{equation}
\Pr(I_b, I_a | h_1)\approx\int_{I_M}S(I_b| I_M)\Pr(I_M|I_a)\Pr(I_a)\ \mathrm{d}I_M\propto \int_{I_M}S(I_b|I_M)\ \mathrm{d}I_M
\end{equation}

We do not know the form of $S$, and so cannot compute this integral
analytically. We cannot rely on a simple Monte Carlo simulation to
estimate it, either: because mental rotation is costly in terms of
cognitive resources and must be performed in a sequence, we cannot
evaluate $S(I_b|I_M)$ at arbitrary $I_M$. Instead, we must choose a
relatively small, sequential set of $I_M$ to estimate the
integral. 

Bayesian quadrature \cite{Diaconis:1988uo} provides an alternate
method for evaluating the integral by placing a prior distribution on
functions and then computing a posterior over values of the
integral. While Bayesian quadrature itself does not address the issue
of choosing points to evaluate the function at, recent work in
statistics and machine learning has examined how to optimally select
these samples \cite{Osborne:2012tm}.

In this project, I will apply these methods to the classic mental
rotation task in both two and three dimensions. From the extensive
literature on mental imagery, I will determine appropriate cognitive
limitations and compute the optimal tradeoff between computation time
(precision) and certainty of answer (accuracy). Finally, I will
compare the results of this model with real behavioral data to test
whether people are, in fact, using a rational solution in the mental
rotation task.


% \renewcommand\bibsection{\subsubsection*{\refname}}
\renewcommand\refname{\normalsize{References}}
\bibliographystyle{ieeetr}
\bibliography{references}
% - References. Give at least three references cited in describing the
% background to your project.

\end{document}



